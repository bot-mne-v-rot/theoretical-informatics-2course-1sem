\subsection{Сведение вопроса выполнимости к вопросу выполнимости в КНФ}
\begin{theorem}
    Существует полиномиальный по времени алгоритм $A$ для любой формулы $\varphi$,
    $A(\varphi)$ -- формула в КНФ и $\varphi$ выполнима $\Longleftrightarrow A(\varphi)$ выполнима. 
\end{theorem}
\begin{proof} \quad

    \begin{center}
        \begin{tikzcd}
            &   &                                           & \boxed{\land}^{\; \textcolor{blue}{t_1}} \arrow[ld, no head] \arrow[rd, no head] &                                          &   \\
            &   & \boxed{\lor}^{\; \textcolor{blue}{t_2}} \arrow[lld, no head] \arrow[d, no head] &                                           & \boxed{\lor}^{\; \textcolor{blue}{t_3}} \arrow[d, no head] \arrow[rd, no head] &   \\
          x_1 &   & \boxed{\land}^{\; \textcolor{blue}{t_4}} \arrow[ld, no head] \arrow[rd, no head] &                                           & \boxed{\lnot}^{\; \textcolor{blue}{t_5}} \arrow[d, no head]                     & x_2 \\
            & x_2 &                                           & \boxed{\lnot}^{\; \textcolor{blue}{t_6}} \arrow[d, no head]                      & \boxed{\lor}^{\; \textcolor{blue}{t_7}} \arrow[d, no head] \arrow[rd, no head] &   \\
            &   &                                           & x_1                                         & x_1                                        & x_3
          \end{tikzcd}
    \end{center}
    Строим систему уравнений по булевой схеме. 
    \begin{gather*}
        \begin{cases}
            \textcolor{blue}{t_1} = \textcolor{blue}{t_2} \land \textcolor{blue}{t_3} \quad \textcolor{blue}{\longrightarrow \text{КНФ}} \\
            \textcolor{blue}{t_2} = x_1 \lor \textcolor{blue}{t_4} \quad \textcolor{blue}{\longrightarrow \text{КНФ}} \\
            \textcolor{blue}{t_3} = \textcolor{blue}{t_5} \lor \textcolor{blue}{t_2} \quad \textcolor{blue}{\dots} \\
            \textcolor{blue}{t_4} = x_2 \land \textcolor{blue}{t_6} \\
            \textcolor{blue}{t_5} = \lnot \textcolor{blue}{t_7} \\
            \textcolor{blue}{t_6} = \lnot x_1 \\
            \textcolor{blue}{t_7} = x_1 \lor x_3 \\
            \textcolor{blue}{t_1} = 1
        \end{cases}
    \end{gather*} 
    Исходная формула выполнима $\Longleftrightarrow$ получившаяся система имеет решение. 
    Хотим смотреть на выполнимости данной формулы как на вопрос выполнимости некой КНФ. Заметим, что, по сути, 
    мы можем вместо фигурной скобки системы подставить дизъюнкцию. Осталось сделать все уравнения системы конъюнктами. Это тоже 
    делается очевидным образом. В уравнении не более трех переменных, значит кнф будет размера не более $2^3$, что есть константа. 

    Мы проверили, что мы можем за нормальное время построить КНФ хорошего вида для нашей формулы. 
    Зачем мы это сделали? Мы свели вопрос выполнимости формулы абстрактного вида к вопросу выполнимости формулы в КНФ. Это гораздо более приятная форма. 
\end{proof}

Теперь рассмотрим различные методы доказательства невыполнимости в КНФ. 
\subsection{Деревья решений для доказательства невыполнимости}
Рассмотрим на примере доказательства невыполнимости следующей системы: 
\begin{gather*}
    \begin{cases}
        x = y \\
        y = z \\
        x \neq z
    \end{cases}
\end{gather*} 
Построим КНФ:
\begin{gather*}
    (x \lor \bar y) \land (\bar x \lor y) \land (y \lor \bar z) \land (z \lor \bar y) \land (x \lor z) \land (\bar x \lor \bar z)
\end{gather*}
Построим дерево решений для этой КНФ. В листьях дерева будут находиться клозы, которые не выполняются при подстановке значений переменных, которые находятся  
на пути от корня до этого листа. 
\begin{center}
    \begin{tikzcd}
        &                                    &                                    & x \arrow[ld, "0"'] \arrow[rd, "1"] &                                    &   \\
        &                                    & y \arrow[ld, "0"'] \arrow[rd, "1"] &                                    & z \arrow[d, "0"'] \arrow[rd, "1"]  &   \\
        & z \arrow[ld, "0"'] \arrow[rd, "1"] &                                    & (x \lor \lnot y)                                  & y \arrow[ld, "0"'] \arrow[rd, "1"] & (\lnot x \lor \lnot z) \\
      (x \lor z) &                                    & (y \lor \lnot z)                                  & (\lnot x \lor y)                                  &                                    & (z \lor \lnot y)
      \end{tikzcd}
\end{center}
Важно, что мы не сможем построить такое дерево для выполнимой КНФ, так как тогда у нас не во всех листьях будут клозы. 

Метод неплохой, иногда может быть полезен для ручной проверки невыполнимости относительно не длинных формул. 
Рассмотрим более хороший метод, который лежит в основе нынешних SAT-солверов(CDCL солверов). 
\subsection{Метод резолюций}
Метод резолюций позволяет доказывать невыполнимость, последовательно выводя из уже имеющихся дизъюнктов новые. 
Пусть у нас есть дизъюнкты:
\begin{gather*}
    C_1 \land C_2 \land \dots \land C_m
\end{gather*}
Раз формулу мы сейчас рассматриваем как множество дизъюнктов, то можно мысленно заменять конъюнкцию на запятые. 
Так вот правило резолюции утверждает следующее: 
\begin{center}
    \AxiomC{$(x \lor A)$}
    \AxiomC{$(\bar x \lor B)$}
    \BinaryInfC{$(A \lor B)$}
    \DisplayProof
\end{center}
Если некоторая интерпретация выполняет обе посылки сверху, то выполняется заключение снизу. 

Что же за загадочная горизонтальная черта? Это самая сложная задача тысячелетия. 
\begin{conj}
    $\varphi$ -- формула в КНФ. Резолюционное поровержение $\varphi$ -- это последовательность $D_1, D_2, \dots, D_s$, где $\forall i: D_i$ -- дизъюнкт, такой, что:
    \begin{gather*}
        \bcases{D_i \text{ -- дизъюнкт } \varphi \cr \frac{D_j, D_k}{D_i} \quad j,k < i}
    \end{gather*}
    То есть каждый дизъюнкт в опровержении -- либо дизъюнкт исходной формулы, либо дизъюнкт, выведенный из двух других.  

    Также необходимым условием корректности опровержения является то, что $D_s$ -- пустой дизъюнкт(тождественная ложь).
\end{conj}

Попробуем доказать невыполнимость нашей формулы с помощью метода резолюций:
\begin{center}
    \AxiomC{$(x \lor \bar y)$}
    \AxiomC{$(y \lor \bar z)$}
    \BinaryInfC{$(x \lor \bar z)$}
    \AxiomC{$\textcolor{blue}{x} \lor z$}
    \BinaryInfC{$x$}
    \AxiomC{$\bar x \lor y$}
    \BinaryInfC{$y$}
    \AxiomC{$z \lor \bar y$}
    \BinaryInfC{$z$}
    \AxiomC{$\bar x \lor \bar z$}
    \BinaryInfC{$\textcolor{blue}{\bar x}$}
    \AxiomC{$x$}
    \BinaryInfC{$\square$}
    \DisplayProof
\end{center}
Давайте докажем два необходимых для любой системы доказательств утверждения:

\underline{Корректность}:
Если $\varphi$ имеет резолюционное опровержение, то $\varphi$ невыполнима.
\begin{proof}
    От противного, пусть $\varphi$ выполнима $\Longrightarrow$ существует интерпретация $I$, которая выполняет $\varphi$. Пусть $D_1, D_2, \dots, D_s$ -- резолюционное
    опровержение $\varphi$. По индукции докажем, что $\forall i \in s[i] \; i$ выполняет $D_i$. 

    База очевидна, переход $i \to i + 1$. На $i+1$-ом шаге у нас два варианта: либо дизъюнкт пренадлежит исходной формуле и тогда очевидно выполняется с помощью $I$, либо был выведен по правилу резолюций. 
    Тогда: 
    \begin{gather*}
        \frac{D_j, D_k}{D_i}, \quad j, k  < i
    \end{gather*}
    По предположению индукции $I$ выполняет $D_j$ и $D_k$. Тогда выполняет и $D_i$. Значит выполняет все клозы. Значит выполняет $D_s$. А это пустой дизъюнкт, который выполнить нельзя. Противоречие. 
\end{proof}
\underline{Полнота}: Если $\varphi$ невыполнима, то существует резолюционное опровержение формулы $\varphi$.

\begin{proof} \quad

    По $\varphi$ построим дерево решений, а по дереву решений построим резолюционное опровержение. 
    Научимся строить резолюционное опровержение по дереву решений. Затребуем от дерева решений следующее условие: хотим, чтобы оно было минимальным. 
    \begin{conj}
        \textbf{Минимальное дерево решений} -- такое дерево решений, что ни один лист не может быть ``поднят выше''. То есть для любого листа верно, что на пути от него к корню все данные о переменных необходимы. 
        Мы не можем забить на что-то и поднять лист выше. 
    \end{conj}
    Итак, приступим к построению резолюционного опровержения. Найдем в дереве решений самый глубокий лист. Утверждается, что у него есть ``брат''. Почему? 
    Мы спустились в этот лист спросив значение какой-то переменной. Их два возможных. Значит есть еще одна веточка дерева, которая идет уже не в этот лист. Почему второй лист 
    будет на той же глубине? Потому что если бы он был ниже, то он бы изначально был самым глубоким. 
    Далее берем этих двух братьев и по правилу резолюций что нибудь из них выводим. Так поднимаемся по дереву. Разберем на примере, который у нас уже был: 
    \begin{center}
        \begin{tikzcd}
            &                                    &                                    & x \arrow[ld, "0"'] \arrow[rd, "1"] &                                    &   \\
            &                                    & y \arrow[ld, "0"'] \arrow[rd, "1"] &                                    & z \arrow[d, "0"'] \arrow[rd, "1"]  &   \\
            & z^{\textcolor{blue}{(x \lor y)}} \arrow[ld, "0"'] \arrow[rd, "1"] &                                    & (x \lor \lnot y)                                  & y^{\textcolor{red}{(\bar x \lor z)}} \arrow[ld, "0"'] \arrow[rd, "1"] & (\lnot x \lor \lnot z) \\
        \textcolor{blue}{\boxed{(x \lor z)}} &                                    & \textcolor{blue}{\boxed{(y \lor \lnot z)}}                                 & \textcolor{red}{\boxed{(\lnot x \lor y)}}                                  &                                    & \textcolor{red}{\boxed{(z \lor \lnot y)}}
        \end{tikzcd}
    \end{center}
    Постепенно поднявшись по дереву срезольвируем два клоза в пустой клоз, который окажется в вершине.
    Почему у нас всегда это получится? Давайте думать. У нас есть два инварианта:
    \begin{enumerate}
        \item Дизъюнкт в вершине опровергается в этой самой вершине. 
        \item Поддерживаем дерево минимальным. 
    \end{enumerate} 
    А из этого очевидно следует, что мы всегда можем построить резолюционное доказательство. 
\end{proof}
\notice \; Легко видеть, что размер резолюционного дерева будет не больше чем размер дерева решений. 

\notice \; Мы построили даже не просто резолюционное доказательство, а древовидное резолюционное доказательство. 

\begin{conj}
    \textbf{Древовидное} резолюционное доказательство -- резолюционное доказательство, в котором каждый клоз используется для вывода единожды. 
\end{conj}

Приведем пример, показывающий, как можно воспользоваться полнотой метода резолюций. Будем проверять выполнимость 2-КНФ с помощью него. Алгоритм следующий:
\begin{itemize}
    \item Пока есть два дизъюнкта, из которых по правилу резолюций можно что-либо новое вывести -- выводим. 
    \item Если выведен пустой дизъюнкт, то выдать, что формула невыполнима. 
    \item Иначе выдать, что она выполнима. 
\end{itemize}  
Корректность алгоритма не вызывает вопросов. Ассимптотика линейная. 

Есть формулы, которые очень невыгодно опровергать с помощью метода резолюций. Например принцип Дирихле. Попытку рассадить $n+1$ голубя в $n$ клеток. 
\begin{theorem}
    Любое резолюционное опровержение $\operatorname{PHP}^{n+1}_n$ имеет размер $2^{\Omega (n)}$.
\end{theorem}
\begin{proof}
    Без доказательства. 
\end{proof}

Докажем аналогичное утверждение для дерева решений. 
\begin{theorem}
    Любое дерево решений для доказательства невыполнимости $\operatorname{PHP}^{n+1}_n$ имеет размер $2^{\Omega (n)}$.
\end{theorem}
\begin{proof} \quad 

    \begin{enumerate}
        \item Сначала докажем, что глубина дерева решений большая. То есть $\Omega (n)$. 
        Докажем это с помощью игры.

        \underline{Игра}: Алиса спрашивает у Боба значение переменной $x$. Боб отвечает значением из $\{0, 1\}$. Боб получает доллар 
        за каждый ответ. Игра заканчивается, когда названные значения переменных опровергают какой-либо дизъюнкт.
        
        \underline{Очевидное утверждение}: Если у Боба есть стратегия, которая позволяет ему заработать $d$ долларов, то глубина любого дерева решений для $\varphi$ не меньше $d$. 

        Осталось придумать стратегию, которая позволит Бобу заработать $n$ долларов в случае с принципом Дирихле. Это нетрудно. Боб может всегда отвечать нулями и тогда Алисе нужно будет 
        задать хотя бы $n$ вопросов, чтобы опровергнуть какой-либо дизъюнкт. 
        \item Теперь мы хотим доказать, что размер дерева будет большим.
        
        Немного модифицируем нашу игру. Теперь Боб может давать три ответа: $0$, $1$ и $*$, в случае которого он дает Алисе самой выбрать значение переменной.
        На этот раз Боб получает доллар только за ответ $*$. 

        \begin{lemma}
            Если у Боба есть стратегия, которая гарантирует ему заработать $d$ долларов, то размер любого дерева решений для $\varphi$ не меньше $2^d$
        \end{lemma}
        \begin{proof}
            Пусть $T$ -- дерево решений. Алиса играет по $T$. Пусть если Боб ответил $*$, то Алиса выбирает значение для переменной случайно равновероятно. Рассмотрим какой-нибудь лист
            $l$. Пусть $P(l)$ -- вероятность того, что мы пришли в лист $l$. Как мы тогда можем ее оценить? Поскольку мы знаем, что спустившись в лист $l$, Боб заработал хотя бы $d$ долларов, 
            то Боб хотя бы $d$ раз отвечал $d$, а Алиса хотя бы $d$ раз кидала монетку. То есть $P(l) \leqslant 2^{-d}$. При этом мы знаем, что: 
            \begin{gather*}
                \sum\limits_{l \text{ -- листы}} P(l) = 1
            \end{gather*}
            А тогда всего листов всего хотя бы $2^d$. Вуаля. 
        \end{proof}
        Все, что нам осталось сделать -- придумать для Боба стратегию заработка. В $\operatorname{PHP}^{n+1}_n$ у нас есть следующие клозы: 
        \begin{itemize}
            \item Дизъюнкты голубей:
            \begin{gather*}
                (p_{i,1} \lor p_{i,2} \lor \dots \lor p_{i,n}) \qquad i \in [n + 1]
            \end{gather*} 
            \item Дизъюнкты клеток:
            \begin{gather*}
                (\overline{p_{i,k}} \lor \overline{p_{j,k}}) \qquad i,j \in [n + 1], \; k \in [n]
            \end{gather*} 
        \end{itemize}
        Выберем следующую стратегию: 
        Если нас спрашивают про переменную $p_{i,j}$ и при этом $j$-ая клетка занята, то есть $p_{k,j} = 1$, то Боб отвечает $0$. В ином случае он отвечает $*$. Как доказать прибыльность стратегии? 
        На коротком дизъюнкте у нас ничего не порушится, так как Боб специально их контролит. Значит формула может заруиниться только на длинном дизъюнкте. То есть у нас заруинился дизъюнкт следующего вида:
        \begin{gather*}
            (p_{i,1} \lor p_{i,2} \lor \dots \lor p_{i,n})
        \end{gather*}
        Раз он рухнул, то у нас на всех значениях оказались нули. Кто мог поставить нули? Могла Алиса, мог Боб. Фиксируем клетку $p_{i,j}$. Если 
        в нее ноль поставила Алиса, то Боб ей это позволил, то есть получил доллар. Если Боб ставил в клетку доллар, то ситуация его вынудила, то есть некая переменная $p_{k,j}$ была равна 1. А кто поставил туда 1? Алиса, так как Боб вообще не ставит единиц. 
        То есть тогда Боб получил за это доллар. 

        Итого, за каждую клетку Боб получил по доллару. То есть Боб заработал хотя бы $n$ долларов. И то есть размер любого дерева решений для $\operatorname{PHP}^{n+1}_n$ имеет размер $2^n$.
    \end{enumerate}
\end{proof}