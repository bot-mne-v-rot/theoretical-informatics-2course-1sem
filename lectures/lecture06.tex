\subsection{Элиминация кванторов в элементарной теории вещественных чисел}
Будем рассматривать естественную интерпретацию сигнатуры $(\R, =, <, +, \times, 0, 1)$. Весь этот параграф будет посвящен доказательству теоремы о том, что данная интрепретация доспускает элиминацию кванторов. 

Приведем пример: \[ \exists x \, (x^2 + px + q = 0) \Leftrightarrow p^2 - 4q \geqslant 0 \]
Тут мы избавились от кванторов и можем легко понять, верна ли эта формула, просто подставив конкретные выражения $p, q$. Также отметим, что этот квантор действует из $\R^2$ в $\{0, 1\}$.

В доказательстве теоремы нам понадобится такое понятие, как алгебраическое подмножество.

\begin{conj}
    Подмножество $\R^n$ является \textbf{полуалгебраическим}, если оно являестя конечным объединением множеств, каждое из которых задается системой полиномиальных уравнений и строгих неравенств.
\end{conj}

Теперь перейдем непостредственно к доказательству.

\vspace{4mm}

\begin{theorem} (Тарского-Зайденберга) \quad
    В $(\R, =, <, +, \times, 0, 1)$ допустима элиминация кванторов. 
\end{theorem}
\begin{proof}
    Как обычно, будем избавляться от самого внутреннего квантора существования и раскручивать это рассуждение дальше по индукции. Пусть предикат имеет вид: \begin{gather*}
        P: \R^n \to \{0, 1\} \\
        P = \exists x \, (\varphi(x, x_1, \dots, x_n))
    \end{gather*}
    Давайте доказывать, что этот предикат задает полуалгебраическое подмножество. Утверждается, что этого будет достаточно для доказательства теоремы. Ну действительно, по любому полуалгебраическому подмножеству легко построить бескванторную формулу, его задающую. Достаточно каждую систему записать  в виде формулы с конъюнкциями, а потом взять дизъюнкцию по всем таким формулам. Тогда если предикат $P$ будет задавать полуалгебраическое подмножество, то он будет эквивалентен этой бескванторной формуле.

    Как доказывать, что $P$ задает полуалгебраическое подмножество?
    Заметим, что атомарные формулы в $\varphi$ могут иметь следующий вид: \[ p(x, x_1, \dots, x_n) = 0 \qquad p(x, x_1, \dots, x_n) > 0 \]
    $p(x, x_1, \dots, x_n)$ это какой-то многочлен из кольца $\Z[x, x_1, \dots, x_n]$. Заметим, что мы также можем смотреть на это как $(\Z[x_1, \dots, x_n])[x]$, то есть это кольцо многочленов от $x$, где коэффициенты это многочлены от $x_1, \dots, x_n$.

    Выпишем все многочлены, входящие в $\varphi: Q_1, \dots, Q_m \in \Z[x, x_1, \dots, x_n]$. Подставим переменным $x_1, \dots, x_n$ какие-то значения $a_1, \dots, a_n$. У нас получатся $\tilde{Q}_1(x), \dots, \tilde{Q}_m(x)$. Введем еще одно определение.

    \begin{conj}
        Пусть $q_1(x), \dots, q_m(x)$ -- многочлены от $x$, $\alpha_1, \dots, \alpha_t$ -- все корни этих многочленов. Тогда \textbf{диграммой для семейства многочленов от одной переменной} называют таблицу следующего вида: 
        \begin{center}
            \begin{tabular}{ ||c|c|c|c|c|c|c|c|| } 
             \hline
              & $(-\infty, \alpha_1)$ & $\alpha_1$ & $(\alpha_1, \alpha_2)$ & $\alpha_2$ & $\dots$ & $\alpha_t$ & $(\alpha_t, +\infty)$ \\
              \hline
              $q_1$ & - & 0 & + & + & $\dots$ & 0 & - \\
             \hline
             \dots & \dots & \dots& \dots& \dots& \dots& \dots& \dots \\ 
             \hline
             $q_m$ & - & + & + & 0 & $\dots$ & - & - \\
             \hline
            \end{tabular}
        \end{center}
        Строки соответствуют многочленам, столбцы корням, а также промежуткам между последовательными корнями. На переcечении $i$-той строки и $j$-того столбца -- знак $i$-того многочлена на $j$-том промежутке (0 означает корень). Смысл диаграммы в том, что столбцы представляют все возможные комбинации знаков многолченов $q_1, \dots, q_m$ в зависимости от $x$.
    \end{conj}

    Оценим, сколько различных таблиц у нас может получиться при различной фиксации $x_1, \dots, x_n$. Строк у нас всегда $m$, столбцов $\leqslant \sum_{i=1}^m \deg_x Q_i$. Таким образом, таких таблиц конечное число.

    Давайте докажем, что для любой диаграммы $D$ множество: \[ \{ \overrightarrow{a} = (a_1, \dots, a_n) \in \R^n | \text{ мн-ны } Q_i(x, \overrightarrow{a}) \text{ имеют диаграмму } D \} \] полуалгебраическое.

    Утверждается, что если мы докажем этот факт, то мы докажем и нашу теорему. Ну действительно, диграмм у нас конечное число, поэтому, все пространство $\R^n$ разобьется в дизъюнктное объединение полуалгебраических подмножеств. На каждом таком подмножестве предикат $P$ либо верен, либо нет, так как все определяет диаграмма.
    Итого, объединив подмножества для тех диаграмм, которые подходят, мы получим полуалгебраическое подмножество. 

    Таким образом, осталось только доказать этот факт. Оказывается. что его легче доказать, если расширить семейство рассмастриваемых многочленов.
    \begin{lemma}
        Если утверждение верно для семейства многочленов $Q_1, \dots, Q_{m+1}$, то оно верно и для $Q_1, \dots, Q_m$.
    \end{lemma}
    \begin{proof}
        Пусть у нас есть диаграмма для $(m+1)$ многочлена. Как она изменится при удалении $(m+1)$-ого многочлена? Удалится последняя строчка, а также все столбцы, соответствующие $\alpha_i$, где $\alpha_i$ -- корень только $(m+1)$-ого многочлена. Иэ этого следует, что диаграмма $D$ для $m$ многочленов может получиться из одной из диаграмм из конечного подмножества диаграмм для $(m+1)$ многочлена. Обозначим их $D_1, \dots, D_k$. Тогда \begin{gather*}
            \{ \overrightarrow{a} = (a_1, \dots, a_n) \in \R^n \, | \, Q_i(x, \overrightarrow{a}), i = 1..m \; \text{ задают диаграмму } D\} = \\
            = \bigcup_{j=1}^k \{ \overrightarrow{a} = (a_1, \dots, a_n) \in \R^n \, | \, Q_i(x, \overrightarrow{a}), i = 1..(m+1) \; \text{ задают диаграмму } D_j\}
        \end{gather*}
        А конечное объединение полуалгебраических тоже полуалгебраическое.
    \end{proof}
    Теперь надо понять, как правильно расширить множество рассматриваемых многочленов. Для этого определим 4 операции для многочленов из $(\Z[x_1, \dots, x_n])[x]$: \begin{enumerate}
        \item Отбрасывание старшего члена (понижает степень по $x$ на 1).
        \item Взятие старшего коэффициента (именно коэффициента, поэтому степень по $x$ становится равна 0).
        \item Дифференцирование по $x$ (понижает степень по $x$ на 1).
        \item Взятие модифицированного остатка при делении одного многочлена на другой. Остаток называется модифицированным, так как при обычном делении у нас могут возникнуть проблемы из-за того, что коэффициенты это тоже многочлены.  
        
        Чтобы $g(x)$ поделилось на $h(x)$ с остатком, нужно домножить $g(x)$ на $(\beta_s)^l$, где $\beta_s$ -- старший коэффициент многочлена $h(x)$, а $l = \deg_x g - \deg_x h$ (я не до конца понял, зачем это, видимо, чтобы в $g(x)$ степени $x_1, \dots, x_n$ были довольно большими). Таким образом, \[ (\beta_s)^l \cdot g(x) = q(x)\cdot h(x) + r(x)  \]
        получаем, что остатком будет $r(x)$.
    \end{enumerate}

    Докажем теперь факт, связанный с этими 4 операциями.
    \begin{lemma}
        Существует конечное надмножество многочленов $Q_1, \dots, Q_m$, которое замкнуто относительно этих операций.
    \end{lemma}
    \begin{proof}
        Каждый многочлен из этого надмножества может быть получен каким-то количеством применений операций. Очевидно, что количество таких применений не превосходит максимальной степени исходных многочленов, так как при каждой операции степень понижается. Отсюда видно, что таких многочленов конечное число. 
    \end{proof}
    Выберем это замкнутое надмножество: $F = \{p_1, \dots, p_t\}$.
    
    \begin{lemma}
        Пусть $F_0$ -- многочлены степени 0 по $x$ из $F$. Тогда по диаграмме для $F_0$ можно однозначно восстановить диаграмму для всего $F$.    
    \end{lemma}

    Пусть $D$ -- диграмма для $F_0$, $D'$ -- диграмма для $F$. Лемма говорит, что по $D$ однозначно определяется $D'$. Но и по $D'$ очевидным образом определяется $D$, так как $D$ это часть $D'$. Значит, между ними есть взаимо однозначное соответствие. Тогда получаем, что 
     \begin{gather*}
        \{ \overrightarrow{a} | F(x, \overrightarrow{a}) \text{ имеет диаграмму } D' \} - \text{полуалгебраическое} \Leftrightarrow \\ \Leftrightarrow
        \{ \overrightarrow{a} | F_0(x, \overrightarrow{a}) \text{ имеет диаграмму } D \} - \text{полуалгебраическое}
    \end{gather*}
    Но во втором случае полуалгебраичность очевидна, так как многочлены степени 0 не зависят от $x$, следовательно в диаграмме $D$ будет один столбец, следовательно это множество будет описываться как: \[ (g(x_1, \dots, x_n) > 0) \land (g_2(x_2, \dots, x_n) < 0) \land (g_3(x_1, \dots, x_n) = 0) \land \dots \]
    Если в диаграмме $D$ +, то будет знак >, если -, то <, если 0, то =.
    
    Таким образом, эти подмножества будут полуалгебраическими, а значит и для диграммы $D'$ для всего $F$ они будут полуалгебраическими.

    Осталось доказать лемму!

    \begin{proof}
        Пусть у нас есть таблица для множества $F_0$.
        Будем добавлять многочлены в таблицу по индукции, считая, что все многочлены меньшей степени по $x$ уже добавлены. 
        При добавлении очередного многочлена $R$ мы должны: \begin{itemize}
            \item Определить значения на $+\infty$ и $-\infty$. Для этого нам надо посмотреть на старший коэффициент. Это какой-то многочлен от $x_1, \dots, x_n$. По правилу 2 он уже есть в таблице. Если он не 0, то во всех столбцах стоит либо +, либо -. Если наш многочлен четный, то значение на $\pm \infty$ совпадает с этим знаком, иначе на $+ \infty$ совпадает, а на $-\infty$ противоположное. Если он 0, то во всех столбцах стоят 0. Но тогда мы можем просто откинуть старший член и по правилу 1, скопировать значение для нашего многочлена без старшего члена.
            \item Определить значения в старых корнях. Посмотрим на какой-то старый корень ненулевого многочлена (т.е. такого, у которого не во всех столбцах 0). Пусть это многочлен $h(x)$. Тогда давайте поделим на него $R$: \[ \beta_{\delta}^l \cdot R = q(x)\cdot h(x) + r(x)  \]
            $q(x) \cdot h(x)$ очевидно будет равно 0, а по знакам $\beta_{\delta}^l$ и $r(x)$ мы одназначно поймем знак $R$ в этой точке.
            \item Добавить новые корни. Корни бывают кратные и не кратные. Утверждается, что новых кратных корней нет, так как это были бы корни производной. Теперь посмотрим на уже проставленные в таблице для старых корней знаки. Точнее посмотрим на последующие: \begin{itemize}
                \item Пусть у нас идут +, - или -, +. Очевидно, что между ними есть корень. Причем один, так как если бы было 2, то была бы точка экстремума, где производная 0, а это в свою очередь корень производной. 
                \item Поймем, что при других комбинациях корней не будет. Это легко понять, просто рассмотрев все комбинации и убедившись, что если корень есть, то есть и корень производной.
            \end{itemize}
            Таким образом, мы легко добавим новые корни. На предыдущие строки эти новые корни не повлияют, так как мы там просто подразобьем промежутки, про которые все знали.
            \item Добавить значение в промежутках. Но это легко делается в зависимости от значения в точках.
        \end{itemize}
        Таким образом, мы полностью однозначно добавили новую строчку и по индукции доказали лемму.
    \end{proof}
    Все, лемма доказана, а значит, и теорема доказана.
    Ура, товарищи! Всем спасибо, все свободны!
\end{proof}