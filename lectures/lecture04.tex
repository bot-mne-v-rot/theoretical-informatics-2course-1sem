\section{Предикатная логика(логика первого порядка)}
По этой теме можно почитать книги Н.К. Верещагина и А.Х.Шеня -- Языки исчисления. 
\subsection{Предикатные формулы}
\begin{conj}
    Введем множество $\mathcal{F}$ -- множество функциональных символов(для обозначения функций), $\mathcal{P}$ -- множество предикатных символов(для обозначения предикатов).
    $(\mathcal{F}, \mathcal{P})$ -- сигнатура. 
    Функции будут иметь вид:
    \begin{gather*}
        M^n \longrightarrow M
    \end{gather*} 
    Предикаты:
    \begin{gather*}
        M^n \longrightarrow \{0, 1\} 
    \end{gather*}
    Множества будут выглядить следующим образом:
    \begin{gather*}
        \mathcal{F} = \{f^{(2)}, g^{(1)}, h^{(3)}, c^{(0)}\} \\
        \mathcal{P} = \{p^{(1)}, q^{(2 )}, s^{(0)}\} \\
        \Gamma = \{x_1, x_2, \dots\} \text{ -- предметные переменные}
    \end{gather*}
\end{conj}
\begin{conj}
    Введем понятие \textbf{терма}:
    \begin{enumerate}
        \item $x \in \Gamma \Longrightarrow x$ -- терм 
        \item $f^{(k)} \in \mathcal{F}$, $t_1, t_2, \dots, t_k$ -- термы $\Longrightarrow f^{(k)} (t_1, t_2, \dots, t_k)$ -- терм
        \item Множество термов -- это минимальное по включению множество строк, которые удовлетворяют условиям (1) и (2). 
    \end{enumerate}
\end{conj}

\textbf{Примеры.} Используем сигнатуру, объявленную чуть выше: \[ f(x, y), f(x, x), g(f(x, x)), f(x, c^{(0)}) \]

\vspace{3mm}

\begin{conj}
    Пусть $p^{(n)} \in \mathcal{P}, t_1, t_2, \dots, t_n$ -- термы, тогда 
    $p^{(n)} (t_1, t_2, \dots, t_n)$ -- \textbf{атомарная формула}. То есть атомарные(элементарные) формулы -- формулы, получающиеся
    приписыванием предикатного символа.  
\end{conj}
\textbf{Замечание: } для соблюдения формальных равенств, как мы далее увидим, определение атомарных формул расширяют. Константы $\operatorname{true}$ и $\operatorname{false}$ -- тождественная истина и ложь, соответственно -- тоже являются атомарными формулами. Но лектор этот момент опустил.

\begin{conj}
    Введем понятие \textbf{предикатной формулы}:
    \begin{enumerate}
        \item Атомрная формула -- это предикатная формула.
        \item Пусть $\varphi, \psi$ -- предикатные формулы, тогда $\lnot \varphi, \varphi \land \psi, \varphi \lor \psi, \varphi \to \psi$ -- тоже предикатные формулы. 
        \item Пусть $\varphi$ -- предикатная формула, тогда для $x \in \Gamma, \; \forall x \, \varphi, \exists x \, \varphi$ -- тоже предикатные формулы.  
    \end{enumerate} 
\end{conj}

\textbf{Пример.} Используем ту же сигнатуру: \[ \forall x \; \exists y \; (p(f(x, y)) \to s) \]

\vspace{3mm}

\begin{conj}
    Каждая переменная может быть \textbf{связанной} или \textbf{свободной}. Переменная называется связанной, если она находится в области действия квантора и свободной(параметром формулы) иначе.
\end{conj}

\begin{conj}
    Введем понятие \textbf{интерпретации для предикатных формул}. Нам понадобятся некоторые 
    второстепенные понятия. 
    \textbf{Носитель интерпретации} -- любое непустое множество $M$.
    \begin{gather*}
        \forall f^{(k)} \in \mathcal{F} \text{ задаем отображение } f^{(k)}_I: M^k \longrightarrow M \\
        \forall p^{(k)} \in \mathcal{P} \text{ задаем отображение } p^{(k)}_I: M^k \longrightarrow \{0, 1\}
    \end{gather*}
    Оценка -- это отображение $\Gamma \longrightarrow M$. Оценка нужная для свободных переменных.
\end{conj}

\begin{conj}
    Введем понятие \textbf{значения формулы $\varphi$ при интерпретации $I$ и оценке $\pi$}.

    \quad Значение терма $t$ при интерпретации $I$ и оценке $\pi$ зададим теперь следующим образом:
    \begin{enumerate}
        \item $x \in \Gamma \qquad x[I, \pi] := \pi(x)$
        \item $f^{(k)}, \; t_1, t_2, \dots, t_k$ -- термы, тогда:
        \begin{gather*}
            f^{(k)}(t_1, t_2, \dots, t_k) [I, \pi] := f^{(k)}_I (t_1[I, \pi], \dots, t_k[I, \pi])
        \end{gather*}
    \end{enumerate}
    \quad Значение атомарной формулы зададим следующим образом:
    \begin{gather*}
        p^{(k)} (t_1, \dots, t_k)[I, \pi] := p^{(k)}_I (t_1[I, \pi], \dots, t_k[I, \pi])
    \end{gather*}

    \quad Зададим теперь для логических связок и кванторов:
    \begin{enumerate}
        \item \begin{gather*}
            \lnot \varphi [I, \pi] := \lnot (\varphi [I, \pi]), \text{ для остальных связок аналогично}
        \end{gather*}
        \item \begin{gather*}
            \exists x \; \varphi[I, \pi] := \bigvee\limits_{a \in M} \varphi [I, \pi^{x \leftarrow a}] 
        \end{gather*}
        \item \begin{gather*}
            \forall x \; \varphi[I, \pi] := \bigwedge\limits_{a \in M} \varphi [I, \pi^{x \leftarrow a}] 
        \end{gather*}
    \end{enumerate}
\end{conj}

\textbf{Пример.} Используем сигнатуру с $\mathcal{F} = \{\}, \mathcal{P} = \{>\}:$ \[ \forall x \; \exists y \; (y > x) \]
Зададим интерпретацию на множестве $\Z$ и обычный предикат $>$ на ней. Тогда в этой интерпретации формула будет верна. Заметим, что оценка тут не важна, так как нет свободных переменных. Если же мы зададим множество, как $M := \{0, 1\}$, и обычный предикат $>$, то в такой интерпретации формула очевидно будет ложна.


\vspace{3mm}

\begin{conj}
    \textbf{Общезначимые формулы} -- это формулы, которые истинны во всех интерпретациях и при всех оценках. 
    Пример:
    \begin{gather*}
        \forall x \; (P(x) \longrightarrow P(x))
    \end{gather*}
\end{conj}

Накидаем (лютому) еще определений. 
\begin{conj}
    Пусть формула $\varphi$ содержит $k$ свободных перемнных. Тогда если у нас задана просто интерпретация без оценки, то можно сказать, что формула $\varphi$ в этой интерпретации задает предикат вида:
    \begin{gather*}
        \varphi[I] : M^k \longrightarrow \{0, 1\}
    \end{gather*}
    Наоборот, предикат
    \begin{gather*}
        P : M^k \longrightarrow \{0, 1\} 
    \end{gather*}
    называется \textbf{выразимым} в $(\mathcal{F}, \mathcal{P})$, если $\mathcal{P}$ можно задать формулой с $k$ параметрами. 
\end{conj}

\vspace{3mm}

\begin{conj}
    \textbf{Арифметика}:
    \begin{align*}
        \mathcal{N} &= \{0, 1, 2, \dots\} \\
        \mathcal{P} &= \{=\} \\
        \mathcal{F} &= \{+, \times \}
    \end{align*}
\end{conj}
Примеры выразимых в арифметике предикатов:
\begin{enumerate}
    \item $x = 0 \qquad x+x=x$
    \item $x=1 \qquad x \times x = x \land \lnot (x=0)$
    \item $x \geqslant y \qquad \exists z \; (x = y + z)$
    \item $x = 239$:
    \begin{gather*}
        \forall y \quad ((y=1) \longrightarrow x = \underbrace{y + y + \dots + y}_{239 \text{ раз}})
    \end{gather*}
    \item $x \mid y \qquad \exists z \; (y = x \times z)$
    \item ``$x$ -- простое число'' \qquad $\forall y \; ((y \mid x) \longrightarrow ((y=1) \lor (y=x)))$
    \item ``$r$ -- остаток при делении $a$ на $b$'' \qquad $\exists q \; (a = qb + r \land r < b)$
    \item ``$x$ -- степень 2'' \qquad $\forall y \; (((y \mid x) \land (y \text{ -- простое})) \longrightarrow ($``$y=2$''$))$
    \item ``$x$ -- степень 4'' \qquad ``$x$ -- степень 2'' $\land \; \exists y : \;(y \times y = x)$
    \item ``$x$ -- степень 6'' \qquad так просто не получится, но Гёдель придумал решение с помощью китайской теоремы об остатках 
\end{enumerate}
\begin{theorem}(Китайская теорема об остатках)

    Пусть $b_1, \dots, b_n$ -- взаимно простые числа. $r_1, \dots, r_n$ -- натуральные числа. Тогда:
        \begin{gather*}
            \exists A : \quad A \equiv r_i \operatorname{mod} b_i
        \end{gather*}
\end{theorem}

Заведем лемму, которая нам поможет выдавать много взаимно простых чисел.

\textbf{Лемма 1.}
    $\forall k \in \N$ существует сколь угодно большое $b \in \N$, т.ч. числа $b+1, 2b+1, 3b+1, \dots, kb+1$ взаимно просты. 

\begin{proof}
    $\gcd((ib + 1), (jb + 1))$ обязан делить разницу между ними: \[ \gcd((ib + 1), (jb + 1)) \mid (j - i)b \]
    $(ib + 1)$ и $(jb + 1)$ взаимно просты с $b$, поэтому: \[ \gcd((ib + 1), (jb + 1)) \mid (j - i) \]
    Возьмем $b = C * k!$ при  любом $C \in \N$. Оно будет подходить, так как $(j - i)$ будет делить $b$, а как мы поняли $\gcd((ib + 1), (jb + 1))$ не будет делить $b$.
\end{proof}

\vspace{3mm}

Применим ее в другой лемме.

\textbf{Лемма 2.}
    $\forall x_1, \dots, x_n \in \N \quad \exists a,b \in \N$:
   \begin{gather*}
       x_i = a \operatorname{mod} (bi + 1)
   \end{gather*}

\begin{proof}
    Лемма 1  + китайская теорема об остатках. 
\end{proof}

Теперь с помощью леммы 2 мы можем закодировать любую конечную последовательность натуральных чисел. Например, выразим предикат "$x$ -- степень 6". Для этого с помощью $a$, $b$ и $k$ закодируем последовтельность, т.ч. $x_0 = 1, x_k = x$ и $\forall i < k \; x_{i + 1} = 6 \cdot x_i$: \[ \exists k \; \exists a, b \; \textbf(a \operatorname{mod} (b + 1) = 1 \textbf) \land \textbf(a \operatorname{mod} (bk + 1) = x \textbf) \land \textbf(\forall i \; ((i < k) \to (a \operatorname{mod} (b(i+1) + 1) = 6 \cdot a \operatorname{mod} (bi + 1))) \textbf) \]

Это довольно сильный и универсальный метод.