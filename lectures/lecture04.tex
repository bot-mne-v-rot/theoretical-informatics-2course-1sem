\section{Предикатная логика(логика первого порядка)}
По этой теме можно почитать книги Н.К. Верещагина и А.Х.Шеня -- Языки исчисления. 
\subsection{Предикатные формулы}
\begin{conj}
    Введем множество $\mathcal{F}$ -- множество функциональных символов(для обозначения функций), $\mathcal{P}$ -- множество предикатных символов(для обозначения предикатов).
    $(\mathcal{F}, \mathcal{P})$ -- сигнатура. 
    Функции будут иметь вид:
    \begin{gather*}
        M^n \longrightarrow M
    \end{gather*} 
    Предикаты:
    \begin{gather*}
        M^n \longrightarrow \{0, 1\} 
    \end{gather*}
    Множества будут выглядить следующим образом:
    \begin{gather*}
        \mathcal{F} = \{f^{(3)}, g^{(1)}, h^{(2)}, c^{(0)}\} \\
        \mathcal{P} = \{p^{(2)}, q^{(0)}, s^{(1)}\} \\
        \Gamma = \{x_1, x_2, \dots\} \text{ -- предметная переменная}
    \end{gather*}
\end{conj}
\begin{conj}
    Введем понятие \textbf{терма}:
    \begin{enumerate}
        \item $x \in \Gamma \Longrightarrow x$ -- терм 
        \item $f^{(k)} \in \mathcal{F}$, $t_1, t_2, \dots, t_k$ -- термы $\Longrightarrow f^{(k)} (t_1, t_2, \dots, t_k)$ -- терм
        \item Множество термов -- это минимальное по включению множество строк, которые удовлетворяют условиям (1) и (2). 
    \end{enumerate}
\end{conj}

\textbf{Примеры:}
% TODO

\begin{conj}
    Пусть $p^{(n)} \in \mathcal{P}, t_1, t_2, \dots, t_n$ -- термы, тогда 
    $p^{(n)} (t_1, t_2, \dots, t_n)$ -- \textbf{атомарная формула}. То есть атомарные(элементарные) формула -- формулы, получающиеся
    приписыванием предикатного символа.  
\end{conj}

\begin{conj}
    Введем понятие \textbf{предикатной формулы}:
    \begin{enumerate}
        \item Атомрная формула -- это предикатная формула
        \item Пусть $\varphi, \psi$ -- предикатные формулы, тогда $\lnot \varphi, \varphi \land \psi, \varphi \lor \psi$ -- тоже предикатные формулы. 
        \item Пусть $\varphi$ -- предикатная формула, тогда для $x \in \Gamma, \; \forall x \varphi, \exists x \varphi$ -- тоже предикатные формулы.  
    \end{enumerate} 
\end{conj}

\begin{conj}
    Каждая переменная может быть \textbf{связанной} или \textbf{свободной}. Переменная называется связанной, если она находится в области действия квантора и свободной(параметром формулы) иначе.
\end{conj}

\begin{conj}
    Введем понятие \textbf{интерпретации для предикатных формул}. Нам понадабятся некоторые 
    второстепенные понятия. 
    \textbf{Носитель интерпретации} -- любое непустое множество. 
    \begin{gather*}
        \forall f^{(k)} \in \mathcal{F} \textbf{ задаем отображение } M^k \longrightarrow M \\
        \forall p^{(k)} \in \mathcal{P} \textbf{ задаем отображение } M^k \longrightarrow \{0, 1\}
    \end{gather*}
    Оценка -- это отображение $\Gamma \longrightarrow M$. 
\end{conj}

\begin{conj}
    Введем понятие значения формулы $\varphi$ при интерпретации $I$ и оценке $\pi$:

    Значение терма $t$ при интерпретации $I$ и оценке $\pi$ зададим теперь следующим образом:
    \begin{enumerate}
        \item $x \in \Gamma \qquad x[I, \pi] := \pi(x)$
        \item $f^{(k)}, \; t_1, t_2, \dots, t_k$ -- термы, тогда:
        \begin{gather*}
            f^{(k)}(t_1, t_2, \dots, t_k) [I, \pi] := f^{(k)}_I (t_1[I, \pi], \dots, t_k[I, \pi])
        \end{gather*}
    \end{enumerate}
    Значение атомарной формулы зададим следующим образом:
    \begin{gather*}
        p^{(k)} (t_1, \dots, t_k)[I, \pi] := p^{(k)}_I (t_1[I, \pi], \dots, t_k[I, \pi])
    \end{gather*}

    Зададим теперь для логических связок и квантров:
    \begin{enumerate}
        \item \begin{gather*}
            \lnot \varphi [I, \pi] := \lnot (\varphi [I, \pi]), \text{ для остальных значений очевидно}
        \end{gather*}
        \item \begin{gather*}
            \exists x \; \varphi[I, \pi] := \bigvee\limits_{a \in M} \varphi [I, \pi^{x \leftarrow a}] 
        \end{gather*}
        \item \begin{gather*}
            \forall x \; \varphi[I, \pi] := \bigwedge\limits_{a \in M} \varphi [I, \pi^{x \leftarrow a}] 
        \end{gather*}
    \end{enumerate}
\end{conj}

\begin{conj}
    \textbf{Общезначимые формулы} -- это формулы, которые истинны во всех интерполяциях и при всех оценках. 
    Пример:
    \begin{gather*}
        \forall x \; (P(x) \longrightarrow P(x))
    \end{gather*}
\end{conj}

