\subsection{Теории и модели}
\begin{conj}
    Для сигнатуры $(\mathcal{P}, \mathcal{F})$ произвольное множество замкнутых формул $T$ называется \textbf{теорией}.
\end{conj}
\begin{conj}
    Интепретация $I$ сигнатуры $(\mathcal{P}, \mathcal{F})$ называется \textbf{моделью} теории $T$, если $I$ выполняет все формулы теории $T$.
\end{conj}
Теперь можно переформулировать \textbf{теорему о компактности}: теория $T$ непротиворечива $\Longleftrightarrow$ теория $T$ имеет модель.

Отметим следствие теоремы о компактности.

\begin{follow}
    \textbf{Теорема Лёвенгейма-Сколема.} 
    Если сигнатура не более, чем счетна, то если $T$ совместна (непротворечива), то $T$ имеет не более, чем счетную модель (под счетностью модели понимаем мощность носителя интерпретации).
\end{follow}
\begin{proof}
    В доказательстве теоремы о компактности модель строилась на множестве термов сигнатуры. Раз сигнатура у нас НБЧС, то и модель НБЧС.
\end{proof}
\begin{conj}
    \textbf{Th(I)} -- множество замкнутых формул, которые истинны в интерпретации $I$.
\end{conj}

Будем выписывать разные свойства, заимствованные из пропозициональной логики.

\begin{conj}
    Синтаксическое следование. Пусть $T$ -- теория, $\varphi$ -- любая формула. Тогда \[ T \vdash \varphi \Longleftrightarrow \exists \psi_1, \dots, \psi_k \in T : (\psi_1 \land \dots \land \psi_k) \to \varphi \text{ явл. общезначимой} \]
    По теореме о полноте это эквивалентно тому, что выводима секвенция \[ \psi_1, \dots, \psi_k \mapsto \varphi  \]
    А это в свою очередь эквивалентно тому, что секвенция $\mapsto \varphi$ выводима из аксиом и секвенций $\mapsto \psi_1, \dots, \mapsto \psi_k$. 
\end{conj}

\begin{conj}
    Из теории $T$ семантически следует замкнутая формула $\varphi$, если любая модель $T$ выполняет и $\varphi$. 

    Обозначается, как $T \vDash \varphi$.
\end{conj}

\begin{theorem}
    О полноте в сильной форме. Пусть $T$ -- теория, а $\varphi$ -- замкнутая формула. Тогда \[ T \vdash \varphi \Longleftrightarrow T \vDash \varphi \]
\end{theorem}
\begin{proof} \quad

    $\quad "\Rightarrow":$ мы поняли, что синтаксическое следствие эквивалентно тому, что секвенция $\mapsto \varphi$ выводима из аксиом и секвенций $\mapsto \psi_1, \dots, \mapsto \psi_k$. Заметим, что аксиомы всегда истинны, а модель по определению выполняет $\psi_1, \dots, \psi_k$. По правилам вывода получаем, что и $\varphi$ будет истинна (действительно, ведь так работает вывод секвенций).

    $\quad "\Leftarrow":$ пусть $T \vDash \varphi$. Тогда $(T, \lnot \varphi)$ противоречива. То есть \[ \exists \psi_1, \dots, \psi_k \in T : (\psi_1 \land \dots \psi_k \land \lnot \varphi) - \text{ противоречие } \]
    А это то же самое, что формула $(\psi_1 \land \dots \land \psi_k) \to \varphi$ общезначима. А это по теореме о полноте эквивалентно выводимости.
\end{proof}

Теперь займемся предикатом равенством. Мы хотим, чтобы он интерпретировался понятным нам образом. Для этого введем следующие аксиомы.

\textbf{Аксиомы равенства:}
\begin{enumerate}
    \item Симметричность: $\forall x, y \; (x = y) \to (y = x)$
    \item Рефлексивность: $\forall x \; (x = x)$
    \item Транзитивность: $\forall x, y, x \; ((x = y) \land (y = x)) \to (x = z)$
    \item Согласованность с функциональными символами: \[\forall f^{(k)} \in \mathcal{F} \,\; \forall x_1, \dots, x_k, y_1, \dots, y_k \quad ((x_1 = y_1) \land \dots \land (x_k = y_k)) \to (f(x_1, \dots, x_k) = f(y_1, \dots, y_k)) \]
    \item Согласованность с предикатными символами: \[\forall p^{(k)} \in \mathcal{P} \,\; \forall x_1, \dots, x_k, y_1, \dots, y_k \quad ((x_1 = y_1) \land \dots \land (x_k = y_k)) \to (p(x_1, \dots, x_k) \to p(y_1, \dots, y_k)) \]
\end{enumerate}
\begin{conj}
    Модель называется \textbf{нормальной}, если предикат равенства интерпретируется, как совпадение элементов.
\end{conj}
\begin{theorem}
    Теорема о компактности для нормальных моделей.

    Теория $T$, в которой содержится предикат равенства, имеет нормальную модель $\Longleftrightarrow (T \; \cap $ \{аксиомы равенства\}) непротиворечива.
\end{theorem}
\begin{proof}\quad

    $\quad "\Rightarrow":$ очевидно, ведь если есть модель, то $T$ непротиворечива, а нормальность дает выполнение аксиом.

    $\quad "\Leftarrow":$ по обычной теореме о компактности $T \; \cup $ \{аксиомы равенства\} имеет модель. В ней равенство не обязано быть совпадением элементов, но это точно отношение эквивалетности. Тогда просто построим новую модель на классах эквивалентности по этому отношению (на каждом классе определим новые функции и предикаты). Детали оставлены как упражнение.
\end{proof}
\begin{follow}
    Пусть $T$ -- теория. Тогда если любая конечная подтеория имеет нормальную модель, то и $T$ имеет нормальную модель.
\end{follow}
\begin{proof}
    Просто здравствуй, просто как дела. Лови еще одно упражнение.
\end{proof}

\begin{example}
    Рассмотрим теорию $Th(\mathcal{N}, 0, 1, +, *, =)$. Очевидно, нормальной моделью является $\mathcal{N}$. Покажем, что существует модель, неизоморфная $\mathcal{N}$.

    Ввведем новую константу $c$ и бесконечное число формул вида: \[ \lnot(c = 0), \lnot(c = 1), \lnot(c = 1 + 1), \dots \]
    Покажем, что если объединим формулы из $Th(\mathcal{N}, 0, 1, +, *, =)$ с нашими, то это все еще будет непротиворечивое множество. Рассмотрим любое конечное подмножество, тогда оно включает в себя конечное число неравенств с констанстой $c$, тогда $\mathcal{N}$ будет для него нормальной моделью.

    Согласно следствию и для всего множества будет нормальная модель. По теореме Левенгейма-Сколема мы можем выбрать не более, чем счетную модель. Но очевидно, что эта модель не будет изоморфна $\mathcal{N}$, потому что константа $c$ не равна никакой сумме единичек.
\end{example}

\begin{conj}
    Пусть $T$ -- теория, $\varphi$ -- замкнутая формула. Тогда $\varphi$ -- \textbf{теорема} теории $T$, если $T \vdash \varphi$.
\end{conj}
\begin{conj}
    Пусть $\Gamma$ -- множество теорем теории $T$. Тогда $\Gamma$ -- аксиоматизация $T$, если $\forall \varphi \in T \quad \Gamma \vdash \varphi$.
\end{conj}
\begin{conj}
    Теория $T$ полная, если для любой замкнутой формулы $\varphi$ верно, что либо $T \vdash \varphi$, либо $T \vdash \lnot \varphi$.
\end{conj}

Например, $Th(I)$ -- всегда полная теория. 

\begin{theorem}
    Пусть $T$ -- непротиворечивая полная теория, а также она конечно аксиоматизируема. Тогда $T$ разрешимая, т.е. существует алгоритм, который проверяет, принадлежит ли замкнутая формула теории $T$).
\end{theorem}
\begin{proof}
    Пусть $A$ -- конечная аксиоматизация $T$. Тогда для любой замкнутой $\varphi$ либо $A \vdash \varphi$, либо $A \vdash \lnot \varphi$. Нам нужно понять, для какик формул выполняется $A \vdash \varphi$. Это делается очевидным образом: для всех строк $s$ (будет перебирать их в порядке возрастания длины) будем проверять, является ли $s$ кодом вывода $A \vdash \varphi$ или $A \vdash \lnot \varphi$. Рано или поздно мы поймем, какой вывод имеется.
\end{proof}

\begin{example}
    \begin{enumerate}
        \item $Th(\Z, =, s, 0)$, где $s(x) = x+1$. Мы уже знаем, что это интерпретация допускает элиминацию кванторов. Поэтому эта теория разрешима (проэлиминируем кванторы и честно проверим истинность). Она также очевидно полная. И не является конечно аксиоматизируемой. Последний факт требует отдельного доказательства.
        
        Для начала построим бесконечную аксиоматизацию. Утверждается, что такая (назовем ее $A$) подходит: \begin{enumerate}
            \item Аксиомы равенства.
            \item $\forall x \; \exists y \;\; s(y) = x$
            \item $\forall x, y \;\; (s(x) = s(y)) \to (x = y)$ (честно говоря, выглядит как аксиома равенства)
            \item $\forall x \;\;  \lnot(s(x) = x)$
            \item $\forall x \;\;  \lnot(s(s(x)) = x)$
            \item $\forall x \;\;  \lnot(s(s(s(x))) = x)$
            \item \dots
        \end{enumerate}
        Почему это аксиоматизация? Пусть у нас есть формула $\varphi$ и мы элиминируем в ней кванторы. Получается последовательность $\varphi \sim \psi_1 \sim \psi_2 \sim \dots \sim \psi_k$, где $\psi_k$ либо 1, либо 0. За каждым значком $\sim$ у нас скрывается некоторое семантическое следование. Так вот все такие следования опираются на аксиомы выше.

        Пусть теперь существует какая-то конечная аксиоматизация. Она состоит из теорем, которые могут быть выведены из аксиоматизации $A$. Таким образом, существует конечный набор теорем $A' \subset A$, который является аксиоматизацией. 

        Для него существует $n$, т.ч: 
        \begin{enumerate}
            \item теорема $(\forall x \;\; \lnot(s^{(n)}(x) = x)) \in A'$
            \item $\forall m > n$ теорема $(\forall x \;\; \lnot(s^{(m)}(x) = x)) \notin A'$
        \end{enumerate} 
        Это все выполняется в силу конечности $A'$.
        
        Осталось прийти к противоречию, а именно построить такую интерпретацию $I$, что $I$ будет выполнять $A'$, но не выполнять $A$ (тогда не любая формула из $A$ будет следовать из $A'$, а следовательно $A'$ не будет аксиоматизацией). Например, подходит $I := \mathbb{Z}_{n + 1}$. Вот такое рассуждение.
        
        \item Теория плотного линейного порядка без первого и последнего элемента.
        

        Пусть сигнатура состоит только из предикатных символов $\mathcal{P} = \{ =, \leqslant \}$. Пусть у нас есть следуюшая теория $T$: \begin{enumerate}
            \item Аксиомы равенства.
            \item Аксиома полного порядка: $\forall x, y \;\; (x \leqslant y) \lor (y \leqslant x)$.
            \item Рефлексивность: $\forall x \;\; x \leqslant x$.
            \item Транзитивность: $\forall x, y, z, \;\; ((x \leqslant y) \land (y \leqslant z)) \to (x \leqslant z)$.
            \item Антисимметричность: $\forall x, y \;\; ((x \leqslant y) \land (y \leqslant x)) \to (x = y)$.
            \item Нет первого элемента: $\forall x \; \exists y \;\; (y < x)$.
            \item Нет последнего элемента: $\forall x \; \exists y \;\; (x < y)$.
            \item Плотность: $\forall x, y \;\; (x < y) \to (\exists z \;\; (x < z) \land (z < y))$.
        \end{enumerate}
        Предикат $(<)$  выражается, как $(x \leqslant y) \land \lnot(x = y)$.

        Теперь давайте докажем, что $T$ является конечной аксиоматизацией теории $Th(\mathbb{Q}, =, \leqslant)$.

        \begin{lemma}
            У теории $T$ нет конечных моделей.
        \end{lemma}
        \begin{proof}
            Очевидно, так как нет первого и последнего элемента.
        \end{proof}
        \begin{lemma}
            Все счетные модели теории $T$ изоморфны.
        \end{lemma}
        \begin{proof}
            Пусть у нас есть две счетные модели $M_1$ и $M_2$. Построим между ними изоморфизм.
            Начнем поочередно перечислять элементы $M_1$ и $M_2$:\begin{itemize}
                \item На первом шаге берем первый элемент $M_1$, назовем его $a_1$, и первый элемент $M_2$, назовем его $b_1$. Говорим, что они переходят друг в друга.
                \item На втором шаге берем $a_2$ и сравниваем его с $a_1$. Если меньше, то ищем $b_k < b_1$ (такой точно найдется). И говорим, что $a_2$ и $b_k$ переходят друг в друга. Если $a_2 > a_1$, то аналогично ищем $b_k > b_1$.
                \item На третьем шаге берем $b_2$. Сравниваем с уже выставленными $b_1$ и $b_k$ и в нужном промежутке ищем $a_m$. Говорим, что они переходят друг в друга.
                \item Продолжаем процесс.
            \end{itemize}
            По построению мы рано или поздно перечислим любой элемент $M_1$ и $M_2$, а также полученные порядки будут изоморфны.
        \end{proof}
        \begin{lemma}
            $T$ является полной теорией.
        \end{lemma}
        \begin{proof}
            От противного: пусть $\exists$ замкнутая $\varphi$, т.ч. $\lnot(T \vdash \varphi) \land \lnot(T \vdash (\lnot \varphi))$. Синтаксическое следование это то же самое, что семантическое, поэтому имеем: \[ \lnot(T \vDash \varphi) \land \lnot(T \vDash (\lnot \varphi)\]
            А это говорит нам о том, что одновременно $(T, \varphi)$ и $(T, \lnot \varphi)$ должны быть совместны (ну пусть, например, $(T, \varphi)$ был бы несовместен, тогда $(T, \lnot \varphi)$ был бы совместен, а это вместе означало бы семантическое следование).

            $(T, \varphi)$ совместен $\Rightarrow$ существует НБЧС модель $M_1$.

            $(T, \lnot \varphi)$ совместен $\Rightarrow$ существует НБЧС модель $M_2$.

            У теории $T$ нет конечных моделей, поэтому $M_1$ и $M_2$ счетны. Но тогда получаем, что $M_1$ и $M_2$ изоморфны, что является противоречием, потому что значение на формуле $\varphi$ будут разными.
        \end{proof}
        \begin{follow}
            $T$ является конечной аксиоматизацией теории $Th(\mathbb{Q}, =, \leqslant)$.
        \end{follow}
        \begin{proof}
            Ну действительно, раз $T$ полная, то либо формула, либо ее отрицание следуют из $T$. В интерпретации $(\mathbb{Q}, =, \leqslant)$ все формулы $T$ истинны, значит, только истинные формулы фогут из нее следовать. А это и есть все формулы $Th(\mathbb{Q}, =, \leqslant)$.
        \end{proof}
        Одновременно еще доказали, что $Th(\mathbb{Q}, =, \leqslant)$ является разрешимой, потому что имеет конечную аксиоматизацию и является полной. Другой способ доказательства разрешимости -- элиминация кванторов.
        \item $Th(\mathcal{N}, =, *, +, 0, 1)$ -- не имеет конечной аксиоматизации.
    \end{enumerate}
    Аксиомы Пеано (нужны просто по приколу): \begin{enumerate}
        \item Аксиомы равенства.
        \item Ассоциативность $+$, дистрибутивность $+$ относительно $*$.
        \item $\forall x \;\; x * 1 = x$.
        \item $\forall x \;\; x + 0 = x$.
        \item Аксиомы индукции для любой формулы $\varphi$ с одним параметром: \[ \varphi(0) \land (\forall n \;\; \varphi(n) \to \varphi(n + 1)) \to (\forall n \;\; \varphi(n)) \]
        
    \end{enumerate}
\end{example}