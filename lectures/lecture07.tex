\subsection{Исчисление секвенций для логики I-го порядка}
\textit{В целом очень похоже на исчисление секвенций для пропозициональной логики,
однако тут добавляются кванторы}

\begin{conj}
    Пусть фиксирована сигнатура $\sigma$. Формула $\varphi$ называется \textbf{общезначимой}, если 
    она истинна в любой интерпретации сигнатуры $\sigma$ на любой оценке 
\end{conj}

Аналогично с пропозциональной логикой определа секвенция $\Gamma \mapsto \Delta$, где $\Gamma,\ \Delta$ - списки предикатных формул

Формульная интерпретация:
\[ \left(\bigwedge_{\varphi \in \Gamma} \varphi \right) \to \left(\bigvee_{\psi \in \Delta} \psi \right)  \]

\begin{conj}
    Контрпример к секвенции $\Gamma \mapsto \Delta$ - пара интерпретация-оценка $(I, \Pi)$, в которых 
    все формулы из $\Gamma$ истинны и все формулы из $\Delta$ ложны.
\end{conj}

\textbf{Утверждение.} $\Gamma \mapsto \Delta$ общезначима $\leftrightarrow \Gamma \to Delta$ не имеет контрпримеров

\textbf{Аксиомы:} $\Gamma \mapsto \Delta : \Gamma \cap \Delta \neq \varnothing$

\textbf{Правила:}
\begin{enumerate}
    \item \AxiomC{$A, B, \Gamma \mapsto \Delta$}
        \UnaryInfC{$(A \land B), \Gamma \mapsto \Delta$}
        \DisplayProof $\quad (\land \mapsto)$ \inlineitem
        \AxiomC{$\Gamma \mapsto \Delta, A$}
        \AxiomC{$\Gamma \mapsto \Delta, B$}
        \BinaryInfC{$\Gamma \mapsto \Delta, (A \land B)$}
        \DisplayProof $\quad (\mapsto \land)$

        \vspace{4mm}

        \item  \AxiomC{$\Gamma \mapsto \Delta, A, B$}
        \UnaryInfC{$\Gamma \mapsto \Delta, (A \lor B)$}
        \DisplayProof $\quad (\mapsto \lor)$ \inlineitem
        \AxiomC{$A, \Gamma \mapsto \Delta$}
        \AxiomC{$B, \Gamma \mapsto \Delta$}
        \BinaryInfC{$(A \lor B), \Gamma \mapsto \Delta$}
        \DisplayProof $\quad (\lor \mapsto)$

        \vspace{4mm}
        
        \item \AxiomC{$\Gamma \mapsto \Delta, A$}
        \UnaryInfC{$\neg A, \Gamma \mapsto \Delta$}
        \DisplayProof $\quad (\neg \mapsto)$ \inlineitem
        \AxiomC{$A, \Gamma\mapsto \Delta$}
        \UnaryInfC{$\Gamma \mapsto \Delta, \neg A$}
        \DisplayProof $\quad (\mapsto \neg)$

        \vspace{4mm}

        \item \AxiomC{$\Gamma \mapsto \Delta, A$}
        \AxiomC{$B, \Gamma \mapsto \Delta$}
        \BinaryInfC{$(A \to B), \Gamma \mapsto \Delta$}
        \DisplayProof $\quad (\to\ \mapsto)$ \inlineitem
        \AxiomC{$A, \Gamma \mapsto \Delta, B$}
        \UnaryInfC{$\Gamma \mapsto \Delta, (A \to B)$}
        \DisplayProof $\quad (\mapsto\ \to)$

        \vspace{4mm}

        \item \AxiomC{$A, \Gamma \mapsto \Delta$}
        \AxiomC{$\Gamma \mapsto \Delta, A$}
        \BinaryInfC{$\Gamma \mapsto \Delta$}
        \DisplayProof $\quad (Cut)$
        
        \vspace{4mm}

        Эти правила мы уже знаем, теперь введём правила для кванторов

        \vspace{2mm}

        \item {\color{red} \AxiomC{$A(t), \Gamma \mapsto \Delta$}
        \UnaryInfC{$\forall x\ A(x), \Gamma \mapsto \Delta$}
        \DisplayProof $\quad (\forall \mapsto)$} \inlineitem
        {\color{blue} \AxiomC{$\Gamma \mapsto A(y), \Delta$}
        \UnaryInfC{$\Gamma \mapsto \forall x\ A(x), \Delta$}
        \DisplayProof $\quad (\mapsto \forall )$}

        \vspace{4mm}

        \item {\color{blue} \AxiomC{$A(y), \Gamma \mapsto \Delta$}
        \UnaryInfC{$\exists x\ A(x), \Gamma \mapsto \Delta$}
        \DisplayProof $\quad (\exists \mapsto)$} \inlineitem
        {\color {red}\AxiomC{$\Gamma \mapsto A(t), \Delta$}
        \UnaryInfC{$\Gamma \mapsto \exists x\ A(x), \Delta$}
        \DisplayProof $\quad (\mapsto \exists )$}

        \vspace{4mm}

        Правила удалений повторений

        \vspace{2mm}

        \item \AxiomC{$\Gamma, A, A \mapsto \Delta$}
        \UnaryInfC{$\Gamma, A \mapsto \Delta$}
        \DisplayProof \inlineitem
        \AxiomC{$\Gamma \mapsto A, A \Delta$}
        \UnaryInfC{$\Gamma \mapsto A, \Delta$}
        \DisplayProof 
\end{enumerate}

Посмотрим на синие правила (11 и 12). В них $y$ - это просто новая переменная, которая не встречается ни в каких других формулах (для избегания коллизий)

Поймём, что если сверху есть контрпример, то и снизу он есть.

11: Предположим существует контрпример, т. е $\Gamma = 1,\ A(y) = 0, \Delta = 0$.
Тогда с этими $\Gamma$ и $\Delta$ выражение $\forall x A(x)$ тоже неверно, и это будет контпримером для секвенции снизу

12: Предположим, существует контпример, т. е. $A(y) = 1,\ \Gamma = 1,\ \Delta = 0$.
Значит при таких $\Gamma$ и $\Delta$ действительно существует $x$, такой что $A(x)$ (в данном случае подошёл $y$). Это и есть 
контпример снизу.

Аналогично, если существует контпример к нижним, значит он является и контрпримером к верхним.
"Синий" случай можно считать простым, так как переход полностью равносилен.\\

Красные правила (10 и 13). В них $t$ - это "свободный для подстановки" терм. 
Свободные переменные определяются естественно по аналогии с Хаскеллем: переменная связана, если её определил некоторый квантор и она находится в его области действия, иначе свободна.

Например, если формула $A(x)$ имеет вид $\varphi(x) \land \exists x\ \psi(x, x)$, то свободным здесь является только $x$ внутри $A(x)$, и подстановку нужно делать только в него.

Что это значит: по аналогии с правилом Cut у нас может оказаться, что если мы подставим вместо $t$ что-то удачное, это позволит нам
доказать общезначимость формулы (см. примеры). Чтобы этим пользоваться, покажем, что у нас есть следствие "сверху вниз"

13: 
Предположим, что есть контпример для нижней формулы. То есть $\Gamma = 1, \Delta = 0$ и не существует $x$, т. ч. $A(x)$.
Тогда при данных $\Gamma$ и $\Delta$ также невозможно выполнить и верхнюю формулу (если мы выполнили A(t), значит значение $t$ могло бы подойти в качестве $x$ снизу).

10: Заметим, что $\forall x A(x)$ выражается как $\neg \exists x (\neg A(x))$, приводя к правилу 13.

\textit{Замечание: обратного следствия нет, и не надо. Если нижняя общезначима, верхняя не обязательно общезначима. Это логично, так как вместо t мы могли подставить
что-то заведомо сломанное}

Пример вывода:
\begin{enumerate}
    \item $\mapsto \forall x P(x) \to \exists x P(x)$
    \item $\forall x P(x) \mapsto \exists x P(x)$
    \item $\forall x P(x) \mapsto P(y)$
    \item $P(y) \mapsto P(y)$
\end{enumerate}

Пример с более сложным выбором терма $t$:
\begin{enumerate}
    \item $\forall x P(x) \mapsto P(f(x)) \land P(g(x, y))$
    \item [2.1] $\forall x P(x) \mapsto P(f(x))$
    \item [2.2] $\forall x P(x) \mapsto P(g(x, y))$
    \item [3.1] $P(f(x)) \mapsto P(f(x))$
    \item [3.2] $P(g(x, y)) \mapsto P(g(x, y))$
\end{enumerate}

\begin{theorem}
    (О полноте и корректности исчисления секвенций)
    $\Gamma \to \Delta$ общезначима $\Leftrightarrow$ $\Gamma \to \Delta$ выводима из аксиом
\end{theorem}

\begin{proof}
    "$\Leftarrow$": следует из корректности правила

    "$\Rightarrow$": Лемма Кёнига - любое бесконечное дерево имеет бесконечную ветвь (очевидно, было на практике).

    Запишем правила {\color{red} (10), (13)} немного по-другому:

    \AxiomC{$A(t), \forall x A(x), \Gamma \mapsto \Delta$}
        \UnaryInfC{$\forall x\ A(x), \Gamma \mapsto \Delta$}
        \DisplayProof $\quad (\forall' \mapsto)$ - (старое правило + удаление повторений)

    \vspace{2mm}
    
    \AxiomC{$\Gamma \mapsto \Delta, \exists x A(x), A(t)$}
    \UnaryInfC{$\Gamma \mapsto \exists x\ A(x), \Delta$}
    \DisplayProof $\quad (\mapsto \exists' )$

    Пусть у нас есть формула $\Gamma \mapsto \Delta$, и мы хотим доказать её общезначимость.
    Про Cut забудем. Все формулы, красных, назовём простыми. Красные заменим на определённые выше и назовём сложными.

    Примерный алгоритм будет такой:
    \begin{enumerate}
        \item Обработать все листься с простыми правилами, пока они есть. Это всегда занимает конечное количество шагов.
        \item Каждому листу со сложным правилом добавить одно "следующее" ответвление
        \item Если остались листья, в которых можно применить какую-то операцию, вернуться к шагу 1.
    \end{enumerate} 

    Тогда у нас будет получаться дерево.

    \begin{tikzcd}
        &                                      &                                      & \Gamma \to \Delta \arrow[lld] \arrow[rrd] &                          &                                      &                                     \\
        & \text{Простая} \arrow[ld] \arrow[rd] &                                      &                                           &                          & \text{Простая} \arrow[ld] \arrow[rd] &                                     \\
    \text{Простая} \arrow[d] &                                      & \text{Простая} \arrow[ld] \arrow[rd] &                                           & \text{Сложная} \arrow[d] &                                      & \text{Простая} \arrow[d] \arrow[ld] \\
    {}                       & {}                                   &                                      & {}                                        & {}                       & {}                                   & {}                                 
    \end{tikzcd}

    Для простых операций понятно, что у вершины будет один или два сына. Как определить шаг для сложной опреации, если 
    вместо $t$ можно подставить любой терм? Будем подставлять по очереди все возможные термы, которые туда подходят. 
    Их счётное количество, поэтому можно (и нужно) гарантировать такой порядок обхода, что любой конкретный терм будет посещён на какой-то конечной глубине 
    (по аналогии с тем, как можно выписать все рациональные числа).

    Переименуем все кванторные переменные, чтобы им соответствовали специальные буквы.

    \textbf{Случай 1}: дерево бесконечно. Тогда есть бесконечная ветвь. Будем по ней строить контрпример.

    Носитель интерпретации - множества термов нашей сигнатуры $+$ переменные из $\Gamma$ (которые не из кванторов)

    Функциональные символы определяются: $f^{(k)}_{I}(t_1, t_2, \cdots, t_k) = f_k(t_1, t_2, \cdots, t_k)$. Можно думать, 
    что мы просто для определили их как .toString(), потому что нам они не очень нужны.

    Предикатные символы: $P_I(t_1, \cdots, t_k)$ - истина, если $P(t_1, \cdots, t_k)$ встречалось слева от стрелки, иначе ложь.
    Если они встречаются и слева, и справа, то заметим, что в какой-то момент они были слева и справа одновременно, а значит сработала аксиома.
    Если формула не встречалась, то можно определить любое значение (пусть будет ложное).

    $\pi(x) = x$

    Проверим для любой формулы, что если она слева, то она истинна, если справа, то ложна.

    Для атомарных по нашему определению.

    Далее индукциея по глубине. База - атомарные. Переход: например, если самая внешняя операция - дизъюнкция $A \lor B$ слева от стрелки, 
    то есть 2 ветви, в одной $A$ слева (значит истинна по индукции), в другой $B$ слева (значит истинна по индукции), значит и $A \lor B$ истинна. 
    Аналогично все остальные связки

    Для синих формул: если $\forall$ справа от стрелки, значит в поддереве $A(y)$ справа от стрелки, значит она ложна. Значит $\forall x A(x)$ тоже ложна. 
    Аналогично для $\exists$ слева от стрелки.

    Для сложных формул: $\forall$ слева от стрелки, значит в поддереве встретились все возможные $A(t)$ слева от стрелки. Все они истинны, значит и $\forall x A(x)$ тоже истина.
    
    $\exists$ справа от стрелки: в поддереве $A(t)$ справа от стрелки для всех возможных $t$, они ложны, значит и $\exists x A(x)$ справа от стрелки тоже ложно.
    
    Значит все формулы в $\Gamma$ будут истинны, все формулы в $\Delta$ ложны, то есть мы построили контрпример.\\

    \textbf{Случай 2}: Дерево конечно. Тогда либо во всех листьях аксиомы (значит мы победили), либо найдётся лист, который не является аксиомой.
    В нём слева и справа стоят атомарные формулы (иначе это был бы не лист). Тогда все формулы слева сделаем истинными, все справа ложными, это и будет контпримером к $\Gamma \to \Delta$

    Интуиция такая: понятно, что трудности возникают только со сложными формулами, потому что надо подобрать подходящее $t$. 
    Давайте гарантируем порядок выбора $t$, такой что все возможные термы будут когда-нибудь выбраны. Если среди них есть такой, который 
    доказывает общезначность, то на нём процесс и завершится. Если такого нет, то ветвь как раз и будет бесконечной, потому что никакое $t$ так и не подойдёт.

\end{proof}

\subsection{Вывод из списка формул}

\begin{conj}
    Пусть $\Gamma$ - список замкнутых формул. $\Gamma$ непротиворечив $\Leftrightarrow$ 
    $\forall \varphi_1, \cdots, \varphi_n \in \Gamma$ $\varphi_1 \land \cdots \land \varphi_n$ не является противоречивой 
    $\leftrightarrow \neg(\varphi_1 \land \cdots \land \varphi_n)$ не является общезначимой.
\end{conj}

\begin{conj}
    $\Gamma$ совместен $\Leftrightarrow$ $\exists I$ - интерпретация, т. ч $\forall \varphi \in \Gamma\ \varphi[I] = 1$
\end{conj}

\begin{theorem}
    (о компактности). $\Gamma$ совместен $\Leftrightarrow \Gamma$ непротиворечив.
\end{theorem}

\begin{proof}
    $"\Rightarrow"$ - очевидно.

    $"\Leftarrow"$ вспомним, как мы доказывали это для пропозициональной логики. Мы дополняли список до полного (для любой формулы содержится либо она, либо её отрицание).
    Тут мы сделаем то же самое, но этого будет недостаточно

    $\Gamma$ - полный $\Leftrightarrow$ для любой замкнутой $\varphi$ либо $\varphi \in \Gamma$, либо $\neg \varphi \in \Gamma$


    \begin{lemma}
        $\Gamma$ непротиворечив $\Rightarrow$ $\Gamma, \varphi$ непротиворечив или $\Gamma, \neg \varphi$ непротиворечив.
    \end{lemma}

    \begin{proof}
        Действительно, если оба противоречивы, то можно взять $\psi_1, \dots, \psi_k \in \Gamma$ противоречивые с $\varphi$ и $\sigma_1, \dots, \sigma_n \in \Gamma$ протвиоречивые с $\neg \varphi$.
        Тогда их объединение содержится в $\Gamma$, значит оно непротиворечиво, но тогда оно должно выполнять или $\varphi$, или $\neg \varphi$
    \end{proof}

    \begin{lemma}
        $\Gamma$ непротиворечив $\Rightarrow \exists \Gamma' \supseteq \Gamma : \Gamma'$ - непротиворечив и $\Gamma'$ полный.
    \end{lemma}

    \begin{proof}
        (для счётной сигнатуры). Для произвольной доказывается через лемму Цорна, но нам этого знать не надо.
        В силу счётности можем для каждой формулы воспользоваться предыдущей леммой, добавляя её или её отрицание.
    \end{proof}

    \begin{conj}
        $\Gamma$ называется экзистенциально полным, если $\exists x \varphi \in \Gamma$, то $\exists$ замкнутый терм $t : \varphi(t) \in \Gamma$.
    \end{conj}

    \begin{lemma}
        $\Gamma$ непротиворечив $\Rightarrow \exists \Gamma' \supseteq \Gamma : \Gamma'$ экзистенциально полный и непротиворечивый. 
        При этом сигнатура $\Gamma'$ содержит новые константы.
    \end{lemma}

    \begin{proof}
        Выпишем список формул, для которых условие экзистенциальной полноты не выполняется:
        $\exists x_1 \varphi_1,\ \exists x_2 \varphi_2 \dots$

        Тогда добавим формулы $\varphi_1(c_1), \varphi_2(c_2) \dots$, где $c_1, c_2 \dots$ - новые константы

        Покажем, что непротиворечивость сохраняетя. Предположим, нашёлся набор, для которого

        $\psi_1 \land \dots \land \psi_k \land \varphi(c_{i_1}) \dots \land \varphi(c_{i_l})$ невыполним. Но мы знаем, что набор

        $\psi_1 \land \dots \land \psi_k \land \exists x_{i_1} \varphi(x_{i_1}) \dots \land  \exists x_{i_l} \varphi(c_{i_l})$ выполним в $\Gamma$. Тогда можно значения $c_{i_k}$ выбрать равными
        значениям соотвествующим $x_{i_k}$ (раз нижняя формула выполняется, значит такие существуют). Тогда и верхняя рмула выполнима.
    \end{proof}

\begin{lemma}
    $\Gamma$ - непротиворечив, тогда $\exists \Gamma' \supseteq \Gamma : \Gamma'$ - полный и экзистенциально полный
\end{lemma}

\begin{proof}
    $\Gamma_0 = \Gamma,\ \Gamma_1 \supseteq \Gamma$ - полное, $\Gamma_2 \supseteq \Gamma_1$ - экзистенциально полное. Но когда мы
    расширили $\Gamma_1$ до $\Gamma_2$ могла пропасть полнота. Тогда будем по очереди расширять, восстанавливая то полноту, то экзистенциальную полноту, получив последовательность

    $\Gamma \subseteq \Gamma_1 \subseteq \Gamma_2 \subseteq \Gamma_3 \subseteq \Gamma_4 \subseteq \cdots$, где чередуются полные и экзистенциально полные.

    $\Gamma' := \overset{\infty}{\underset{i = 0}{\bigcup}} \Gamma_i$ - полное и экзистенциально полное, так как любая формула попадет в какой-то конкретный $\Gamma_i$.
\end{proof}

\begin{lemma}
    $\Gamma'$ - полный, экзистенциально полный и непротиворечивый, тогда $\exists$ интепретация
    $I$, такая что $I$ выполняет все формулы из $\Gamma'$ 
\end{lemma}

\begin{proof}
    Носитель $I$ - множество замкнутых термов.

    Функциональные символы: $f_I(t_1, \dots, t_k) = f(t_1, \dots, t_k)$ - просто приписали функциональный символ к термам, а-ля .toString()

    Предикатные: $P_I(t_1, \dots, t_k)$ - истина, если $P_I(t_1, \dots, t_k) \in \Gamma'$, и ложно, если $\neg P_I(t_1, \dots, t_k) \in \Gamma'$

    Индукцией по построению формулы доказываем $\varphi[I] = 1 \Leftrightarrow \varphi \in \Gamma$, $\varphi[I] = 0 \Leftrightarrow \neq \varphi \in \Gamma$

    $\land, \lor, \neg, \Rightarrow$ доказывается очевидно аналогично пропозициональному

    Для формулы $\exists x \varphi(x)$: 
    Хотим проверить, что 
    $\begin{cases}
        \varphi \in \Gamma \Rightarrow \varphi[I] = 1\\
        \varphi \notin \Gamma \Rightarrow \varphi[I] = 0
    \end{cases}$

    Пусть $\exists x \varphi(x) \in \Gamma$. Тогда по экзистенциальной полноте для некоторого терма $t\ \varphi(t) \in \Gamma \Rightarrow \varphi(t)[I] = 1 \Rightarrow \exists x \varphi(x)[I] = 1$

    $\exists x \varphi(x) \notin \Gamma \Rightarrow \neg \exists x \varphi \in \Gamma \Rightarrow \forall t \varphi(t) \notin \Gamma$. По ИП для любого $t$ $\varphi(t)$ ложна
    $\Rightarrow \exists x \varphi[I] = 0$

    Аналогично для $\forall$

    $\forall x \varphi \in \Gamma$. Хотим, чтобы для любого $t$ $\varphi(t) \in \Gamma$.
    Допустим, это не так. Тогда по полноте $\neg \varphi t \in \Gamma$. Но тогда $\forall x \varphi \land \neg \varphi(t)$ - противоречие 

    $\forall x \varphi \notin \Gamma \Rightarrow \neg \forall x \varphi \in \Gamma \Leftrightarrow \exists x(\neg \varphi) \in \Gamma \Rightarrow$ по экзистенциальной полноте
    найдется $t$, т. ч. $\neg \varphi(t) \in \Gamma$. Тогда $\neg \varphi(t)[I] = 1 \Rightarrow \forall x \varphi[I] = 0$
\end{proof}

Объединяя 4 леммы получаем совместность.

\end{proof}