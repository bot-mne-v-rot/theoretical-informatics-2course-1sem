\subsection{Метод автоморфизмов}

\begin{conj}
    Пусть $I$ --- интерпретация сигнатуры $(\mathcal{F}, \mathcal{P})$, $M$ --- носитель этой интерпретации. Биекция $\alpha : M \to M$ называется \textbf{автоморфизмом}, если:
    \begin{enumerate}
        \item Любой $p^{(k)} \in \mathcal{P}$ устойчив относительно $\alpha$, т.е.:
        $$ p^{(k)}_{I} (\alpha(m_1), \alpha(m_2), \dots, \alpha(m_k)) = p^{(k)}_{I}(m_1, m_2, \dots, m_k)$$
        \item Любая $f^{(k)} \in \mathcal{F}$ устойчива относительно $\alpha$, т.е.:
        $$ f^{(k)}_I (\alpha(m_1), \alpha(m_2), \dots, \alpha(m_k)) = \alpha(f^{(k)}_I(m_1, m_2 , \dots, m_k))$$
    \end{enumerate} 
\end{conj}

\begin{theorem}
    Если предикат $Q$ выразим в интерпретации $I$, то $Q$ устойчив относительно автоморфизмов $I$.
\end{theorem}
\textbf{Пример.} \\
Рассмотрим $(\Z, =, +)$. Покажем, что предикат $x < y$ невыразим.
Рассмотрим $\alpha(x) = -x$. Это автоморфизм, т.к.:
\begin{itemize}
    \item $-m = -n \Leftrightarrow m = n$
    \item $(-m) + (-n) = -(m + n)$
\end{itemize}
Но $a < b \nRightarrow {-a < -b}$.
\begin{proof}
    Пусть $\pi : X \to M$ --- оценка.
    \begin{enumerate}
        \item Пусть $t$ --- терм. Докажем, что $t[I, \alpha \circ \pi] = \alpha(t[I, \pi])$.
        \begin{itemize}
            \item Если $t = x \in \Gamma$ --- переменная, то это просто по определению.
            \item Если $t = f^{(k)}(t_1, t_2, \dots, t_k)$, то воспользуемся одной из аксиом автоморфизма:
            \begin{align*}
                f^{(k)}(t_1, t_2, \dots, t_k)[I, \alpha \circ \pi] &= 
                f^{(k)}_I(t_1[I, \alpha \circ \pi], t_2[I, \alpha \circ \pi], \dots, t_k[I, \alpha \circ \pi]) \\
                &= f^{(k)}_I(\alpha(t_1[I, \pi]), \alpha(t_2[I, \pi]), \dots, \alpha(t_k[I, \pi])) \\
                &= \alpha(f^{(k)}_I(t_1[I, \pi], t_2[I, \pi], \dots, t_k[I, \pi])) \\
                &= \alpha(f^{(k)}_I(t_1, t_2, \dots, t_k)[I, \pi]) \\
            \end{align*}
            Т.е. просто по определениям вытаскиваем $\alpha$ наружу.
        \end{itemize}
        \item Пусть $\varphi$ --- атомарная формула. Докажем, что $\varphi[I, \alpha \circ \pi] = \varphi[I, \pi]$ аналогично предыдущему пункту:
        \begin{align*}
            p^{(k)}(t_1, t_2, \dots, t_k)[I, \alpha \circ \pi] &= 
            p^{(k)}_I(t_1[I, \alpha \circ \pi], t_2[I, \alpha \circ \pi], \dots, t_k[I, \alpha \circ \pi]) \\
            &= p^{(k)}_I(\alpha(t_1[I, \pi]), \alpha(t_2[I, \pi]), \dots, \alpha(t_k[I, \pi])) \\
            &= p^{(k)}_I(t_1[I, \pi], t_2[I, \pi], \dots, t_k[I, \pi]) \\
            &= p^{(k)}_I(t_1, t_2, \dots, t_k)[I, \pi] \\
        \end{align*}
        \item Пусть наша формула --- это $\lnot \varphi$, $\varphi \vee \psi$, $\varphi \wedge \psi$ или $\varphi \to \psi$. Тут всё достаточно просто проверяется.
        \item Пусть наша формула имеет вид $\exists x \; \varphi$. Докажем, что $\exists x \; \varphi[I, \alpha \circ \pi] = \exists x \; \varphi[I, \pi]$.
        \begin{align*}
            \exists x \; \varphi[I, \alpha \circ \pi] = \bigvee_{a \in M} \varphi[I, \alpha(\pi)^{x \leftarrow a}]
        \end{align*}
        Т.к. $\alpha$ -- биекция, то можно заменить подстановку $x \leftarrow a$ на $x \leftarrow \alpha(a)$ ($\alpha(a)$ так же ''пробегает'' все значения). Тогда:
        \begin{align*}
            \exists x \; \varphi[I, \alpha \circ \pi] &= \bigvee_{a \in M} \varphi[I, \alpha(\pi)^{x \leftarrow a}] \\
            &= \bigvee_{a \in M} \varphi[I, \alpha(\pi)^{x \leftarrow \alpha(a)}] \\
            &= \bigvee_{a \in M} \varphi[I, \alpha(\pi^{x \leftarrow a})] \\
            &= \bigvee_{a \in M} \varphi[I, \pi^{x \leftarrow a}] \\
            &= \exists x \; \varphi[I, \pi]
        \end{align*}
        \item Для квантора всеобщности ($\forall$) все рассуждения будут точно такими же.
    \end{enumerate}
\end{proof}
\notice Нам достаточна только сюръективность автоморфизма.

\textbf{Примеры:}
\begin{enumerate}
    \item $(\Z, =, <)$. Предикат $x = 0$ невыразим. Подходит автоморфизм $x \mapsto x + 1$.
    \item $(\Q, =, <, +)$. Предикат $x = 1$ невыразим. Подходит автоморфизм $x \mapsto \frac{x}{2}$. \\
    Предикат $x = 0$ выразим: $x = 0 \Leftrightarrow x + x = x$.
    \item $(\R, =, <, 0, 1)$. Предикат $x = \frac{1}{2}$ невыразим. Подходит автоморфизм $x \mapsto x^3$. 
    \item $(\R, =, +, 0, 1)$. Все рациональные числа выразимы.
    \begin{align*}
        x=\frac{m}{n} &:  &\underbrace{x+x+...+x}_{n} = \underbrace{1+1+...+1}_{m} \\
        x=-\frac{m}{n} &:  &\underbrace{x+x+...+x}_{n} + \underbrace{1+1+...+1}_{m} = 0\\
    \end{align*}
    Докажем, что если $\gamma \notin \Q$, то $\gamma$ невыразимо. Посмотрим на $\R$, как на линейное пространство над $\Q$, т.е. $\R$ -- векторы, $\Q$ -- скаляры. Пусть $\gamma_1 \neq \gamma_2$, $\gamma_1, \gamma_2 \notin \Q$. Тогда $(1, \gamma_1)$ -- ЛНС и $(1, \gamma_2)$ -- ЛНС. Их можно продолжить до (бесконечного) базиса.
    \begin{align*}
        & (1, \gamma_1) \text{  продолжим до  } (1, \gamma_1, e_1, e_2, \dots) \\
        & (1, \gamma_2) \text{  продолжим до  } (1, \gamma_2, e_1', e_2', \dots) \\
    \end{align*}
    Введём отображение $\alpha : \R \to \R$. Пусть $x = \alpha_1 \cdot 1 + \alpha_2 \gamma_1 + \alpha_3 e_1 + \dots$ Тогда $\alpha(x) = \alpha_1 \cdot 1 + \alpha_2 \gamma_2 + \alpha_3 e_1' + \dots$ Т.е. мы берём число, раскладываем его по первому базису, затем те же самые координаты подставляем под второй базис. Проверим, что это автоморфизм:
    \begin{itemize}
        \item $\alpha(0) = 0$, т.к. базис ЛНС, а значит все коэф. равны 0.
        \item $\alpha(1) = 1$ в силу единственности разложения и того, что 1 присутствовала и там, и там на одинаковых позициях.
        \item $x = y \Rightarrow \alpha(x) = \alpha(y)$. Действительно,
        в первом базисе у $x$ и $y$ были одинаковые координаты и во втором будут одинаковые.
        \item $\alpha(x) + \alpha(y) = \alpha(x + y)$ очевидно.
    \end{itemize}
    Пусть был предикат $x = \gamma_1$. Если подставить $x := \gamma_1$, то предикат будет верен. Но $\alpha(\gamma_1) = \gamma_2 \neq \gamma_1$. Т.е. этот предикат не устойчив относительно $\alpha$.
\end{enumerate} 

\subsection{Элиминация кванторов}
\begin{theorem}
    Любой предикат, выразимый в $(\Z, S, 0, =)$, где $S(x) := x + 1$, выразим и формулой, которая не содержит кванторов.
\end{theorem}
\textbf{Замечание. } Говорят, что $(\Z, S, 0, =)$ допускает элиминацию кванторов.
\begin{proof} $ $
     
    Избавляемся от кванторов по одному. Поймём, что нам достаточно научиться избавляться от квантора существования. Действительно:
    $$ \forall x \; \varphi(x, y_1, \dots, y_n) \Longleftrightarrow \lnot \left(\exists x \; \lnot \varphi(x, y_1, \dots, y_n) \right)$$
    Пусть теперь наша формула имеет вид:
    $$ \exists x \; \underbrace{\varphi(x, y_1, \dots, y_n)}_{\text{бескванторная}} $$
    Нам нужно научиться избавляться от $x$. Внутри $\varphi$ бывают термы следующих видов:
    \begin{itemize}
        \item $S(S(\dots S(x)\dots))$
        \item $S(S(\dots S(y_i)\dots))$
        \item $S(S(\dots S(0)\dots))$
    \end{itemize}
    Тогда атомарные формулы могут быть видов:
    \begin{itemize}
        \item $x = x + C \Leftrightarrow 0 = C$
        \item $x = C$
        \item $x = y + C$
    \end{itemize}
    Из формулы первого вида можно сразу $x$ выкинуть, т.к. от $x$ они не зависят. Константа $C$ может быть отрицательной. Например, предикат $x = -1$ можно выразить, как $S(x) = 0$.
    Выпишем все атомарные формулы, которые встречаются в $\varphi$:
    \begin{align*}
        x = t_1 & \Leftrightarrow x = c_1 \\
        x = t_2 & \Leftrightarrow x = c_2 \\
        & \vdots \\
        x = t_{m-k} & \Leftrightarrow x = c_{m-k} \\
        x = t_{m-k+1} & \Leftrightarrow x = y_{i_1} + b_1 \\
        x = t_{m-k+2} & \Leftrightarrow x = y_{i_2} + b_2 \\
        & \vdots \\
        x = t_m & \Leftrightarrow x = y_{i_k} + b_k \\
    \end{align*}
    Рассмотрим следующую дизъюнкцию:
    $$ \varphi(t_1, y_1, \dots, y_n) \vee \dots \vee \varphi(t_m, y_1, \dots, y_n) $$
    Мы просто подставляем в формулу вместо $x$ каждое его значение, которое где-то используется. 
    
    Нужно быть аккуратным при подстановке, т.к. константы могут быть отрицательными. Пусть $t_i = -3$. И у нас был какая-то атомарная формула $x = S(0)$, тогда заменим её на $0 = S(S(S(S(0))))$.

    Сейчас у нас в каждом клозе хоть одна атомарная формула выполняется. Чтобы получившаяся бескванторная формула была эквивалентна исходной с квантором, нужно добавить формулу $\widetilde{\psi}$, в которой ни одна атомарная формула не выполняется. 

    Пусть $\widetilde{\psi}$ --- формула, полученная из $\varphi$, заменой всех атомарных формул с $x$ на ложь. Например, на $0 = S(0)$.

    Получаем:
    $$ \exists x \; \varphi(x, y_1, \dots, y_n) \Longleftrightarrow \varphi(t_1, y_1, \dots, y_n) \vee \dots \vee \varphi(t_m, y_1, \dots, y_n) \vee \widetilde{\psi}(y_1, \dots, y_n) $$
\end{proof}

\begin{theorem}
    $(\Q, =, <)$ допускает элиминацию кванторов.
\end{theorem}
\begin{proof} 
    Достаточно избавиться от одного квантора. Пусть формула имеет вид: $$\exists x \; \underbrace{\varphi(x, y_1, \dots, y_n)}_{\text{бескванторная}}$$

    Для начала избавимся от импликаций в $\varphi$:
    $a \to b \Leftrightarrow \lnot a \vee b$.

    Теперь избавимся от отрицаний в $\varphi$:
    \begin{itemize}
        \item $\lnot x = y \Leftrightarrow (x < y) \vee (x > y)$
        \item $\lnot x < y \Leftrightarrow (x = y) \vee (y < x)$
        \item В более сложных формулах просто пользуемся законами де Моргана, пока отрицание не дойдёт до атомарных формул.
    \end{itemize}

    Приведём $\varphi$ в ДНФ без отрицаний, где пропозициональные переменные --- атомарные формулы. Т.к. отрицаний и импликаций нет, это можно сделать, аккуратно раскрыв скобки: 
    $$(a \vee b) \wedge (c \vee d) = (a \wedge c) \vee (a \wedge d) \vee (b \wedge c) \vee (b \wedge d) $$
    Получили, что исходная формула имеет вид:
    $$\exists x \; (D_1 \vee D_2 \vee \dots \vee D_m) \Longleftrightarrow \exists x \; D_1 \vee \exists x \; D_2 \vee \dots \vee \exists x \; D_m$$

    Теперь каждая часть дизъюнкции имеет очень простой вид:
    $$ \exists x \bigwedge \begin{sqcases}
        x = \dots \\
        x < \dots \\
        x > \dots \\
        \text{что-то без $x$}
    \end{sqcases}$$
    То, что, не содержит $x$ можно сразу вынести вне области видимости квантора. Более того, в правой части формул-сравнений может быть только другая переменная, т.к. $x = x$ можно сразу заменить на что-то истинное, а $x < x$ на что-то ложное.
    
    Если дизъюнкт содержит $x = \dots$, то можно этот конъюнкт переписать так же, как мы это делали при элиминации кванторов в $(\Z, S, 0, =)$.

    В остальных случаях получаем ещё более простую вещь, которую тоже можно записать без кванторов:
    $$ \exists x \bigwedge \begin{sqcases}
        x < t_1 \\
        \vdots \\
        x < t_k \\
        x > z_1 \\
        \vdots \\
        x > z_l \\
    \end{sqcases}
    \Longleftrightarrow
    \bigwedge_{\substack{1 \leqslant i \leqslant k \\ 1 \leqslant j \leqslant l}} z_j < t_i$$ 
    Эта равносильность верна в силу плотности поля рациональных чисел.
    Тонкий момент: если есть ограничения только снизу или только сверху, то формула тождественно истинна, т.к. мы всегда можем взять достаточно большое или достаточно малое рациональное число.
\end{proof}
\textbf{Главное замечание.} У нас возникает проблема с тем, что в текущих определениях элиминация кванторов в замкнутой формуле невозможна. Т.к. в этой интерпретации нет констант и функций, все термы имеют вид --- какая-то переменная. Тогда атомарные формулы имеют вид $x=y$ и $x<y$. Т.к. формула не может быть пустой, любая бескванторная формула имеет хотя бы одну свободную переменную. Но с помощью кванторов мы можем легко построить замкнутую формулу, т.е. не содержащую свободных переменных. Получаем, что нульместные предикаты нельзя выразить бескванторной формулой. Для того, чтобы это исправить язык расширяют, добавляя константы $\operatorname{true} $ и $\operatorname{false} $. Но лектор этот момент замял.

\textbf{Замечания.} 
\begin{enumerate}
    \item $(\R, =, <)$ также допускает элиминацию кванторов. Просто в точности повторим рассуждения.
    \item Формулы, истинные в $(\R, =, <)$, истинны и в $(\Q, =, <)$. Действительно, избавившись от кванторов одним и тем же алгоритмом, мы получим одинаковые формулы без кванторов. Тогда если у второй есть контрпример, у первой он тоже есть. В обратную сторону, это тоже верно. Говорят, что эти интерпретации \textbf{элементарно эквивалентны}.
    \item Интерпретация $(\Q, =, <, +, \text{рац. константы})$ также допускает элиминацию кванторов. Ход рассуждения здесь точно такой же. Мы всегда можем перенести $x$ в одну сторону и получить ограничения вида:
    $$\begin{sqcases}
        x = \dots \\
        x < \dots \\
        x > \dots \\
        \text{что-то без $x$}
    \end{sqcases}$$
    \item Аналогично, $(\R, =, <, +, \text{рац. константы})$ также допускает элиминацию кванторов.
\end{enumerate}

\textbf{Пример:} \\
Докажем некоторое нетривиальное математическое утверждение, пользуясь тем, что некоторая интерпретация допускает элиминацию кванторов.
\begin{theorem}
    Пусть квадрат $1\times1$ разрезан на несколько квадратов. Тогда длины сторон этих квадратов -- рациональные числа.
\end{theorem}
\begin{proof}
    Пусть $k$ -- это число квадратов в разрезании. Пусть $a_1, a_2, \dots, a_k$ -- длины сторон. Докажем, что если существует разрезание с такими длинами сторон, то эти числа обязаны быть рациональными.

    Построим формулу, использующую переменные: $x_1, y_1, x_2, y_2, \dots, x_k, y_k, a_1, a_2, \dots, a_k$, где $(x_i, y_i)$ -- координаты левого нижнего угла $i$-го квадрата.

    Запишем предикат $\varphi(x_1, y_1, \dots, x_k, y_k, a_1, \dots, a_k)$ о том, что разрезание корректно. Для этого надо записать следующие условия:
    \begin{itemize}
        \item Никакие два квадрата не пересекаются.
        \item Квадраты не вылезают за пределы.
        \item Каждая точка покрыта.
    \end{itemize}
    Не трудно понять, что сигнатуры $(+, =, <, \text{рац. константы})$ хватит (лектор оставил это в качестве упражнения).
    
    Теперь можно записать "существует разрезание со сторонами $a_1, \dots, a_k$":
    $$ \exists x_1, y_1, \dots, x_k, y_k : \varphi(x_1, y_1, \dots, x_k, y_k, a_1, \dots, a_k) $$
    Т.к. есть элиминация кванторов, эту формулу можно записать без кванторов. Без кванторов получается формула $\psi(a_1, \dots, a_k)$. Причём можно считать, что эта формула записана в виде ДНФ без отрицаний. Получаем, что:
    $$\psi(a_1, \dots, a_k) = \bigvee_{i=1}^m D_i$$

    Возьмём $a_1, \dots, a_k$ такие, что $\psi(a_1, \dots, a_k)$ верна. Тогда существует $i_0$ такой, что $D_{i_0}$ верен. Этот конъюнкт имеет вид:
    $$ D_{i_0} = \begin{cases}
        \begin{cases}
            l_1(a_1, \dots, a_k) = \gamma_1 \\
            l_2(a_1, \dots, a_k) = \gamma_2 \\
            \vdots
        \end{cases} \\
        \begin{cases}
            l_1'(a_1, \dots, a_k) > \gamma_1' \\
            l_2'(a_1, \dots, a_k) > \gamma_2' \\
            \vdots
        \end{cases} \\
    \end{cases} $$
    Где $l_i, l_i'$ -- некоторые линейные комбинации.
    Заметим, что система из равенств образует СЛУ. Рассмотрим случаи:
    \begin{itemize}
        \item СЛУ имеет одно решение. Т.к. все коэффициенты рациональны, то по методу Крамера или Гаусса можно понять, что тогда и решение будет тоже рационально.
        \item СЛУ не имеет решений. Такого быть не может, т.к. мы взяли выполнимый конъюнкт.
        \item СЛУ имеет бесконечно много решений. Поймём, что такого быть не может. Множество решений образуют некоторое аффинное подпространство. Значит, найдётся прямая, проходящая через $(a_1, \dots, a_k)$, каждая точка которой является решением СЛУ.
        
        Каждое неравенство высекает на этой прямой некоторый луч, причём раз все неравенства строгие, после пересечения всех этих лучей остаётся некоторое открытое множество решений (неравенств конечное число, конечное пересечение открытых множеств -- открытое множество). Значит, точка $(a_1, \dots, a_k)$ лежит там с некоторой окрестностью. Значит, в нём можно выделить некоторый отрезок, каждая точка которого удовлетворяет обеим системам.

        НУО, одна из границ этого отрезка -- $(a_1, \dots, a_k)$. Этот отрезок можно задать некоторым вектором $(h_1, \dots, h_k)$. Тогда множество точек описывается:
        $$ (a_1 + t h_1, \dots, a_k + t h_k) \text{ при } t \in [0, 1] $$

        Но у нас есть свойство, что дизъюнтное объединение квадратов -- квадрат $1 \times 1$. Таким образом, сумма площадей всех квадратов равна 1:
        $$(a_1 + t h_1)^2 + \dots + (a_k + t h_k)^2 = 1$$
        Но это уравнение второй степени относительно $t$. У него не более двух решений. Противоречие.
    \end{itemize}
\end{proof}

